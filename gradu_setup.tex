
\usepackage[utf8]{inputenc}
\usepackage[english]{babel}
\usepackage{amsthm}
\usepackage{amsfonts}
\usepackage{amsmath}
\usepackage{amssymb}
\usepackage{enumerate}
\usepackage{comment}
\usepackage{xcolor}

\newcommand{\R}{\mathbb{R}}
\newcommand{\C}{\mathbb{C}}
\newcommand{\Q}{\mathbb{Q}}
\newcommand{\N}{\mathbb{N}}
\newcommand{\D}{\mathbb{D}}
\newcommand{\No}{\mathbb{N}_0}
\newcommand{\Z}{\mathbb{Z}}
\newcommand{\F}{\mathbb{F}}
\newcommand{\unitcircle}{\mathbb{S}}
\renewcommand{\H}{\mathcal{H}}
\newcommand{\Hp}{\mathbb{H}_{+}}
\newcommand{\Hpc}{\overline{\mathbb{H}}_{+}}
\newcommand{\diam}{\operatorname{diam}}
\newcommand{\Lip}{\operatorname{Lip}}
\renewcommand{\L}{\mathcal{L}}
\newcommand{\tr}{\text{tr}}
\newcommand{\spec}{\text{spec}}
\renewcommand\atop[2]{\genfrac{}{}{0pt}{}{#1}{#2}}
\newcommand{\eps}{\varepsilon}
\newcommand{\id}{\operatorname{id}}
\newcommand\restr[2]{{% we make the whole thing an ordinary symbol
  \left.\kern-\nulldelimiterspace % automatically resize the bar with \right
  #1 % the function
  \vphantom{\big|} % pretend it's a little taller at normal size
  \right|_{#2} % this is the delimiter
  }}
\newcommand*{\sijoitus}[2]
{\mathop{\Big/}\limits_{\mspace{-18mu}#1}^{\mspace{17mu}#2}}
\newcommand\incl[2]{J_{#2}}
\newcommand\arestr[2]{#1_{#2}}
\newcommand{\toplane}{\xi}
\newcommand{\tocircle}{\eta}
\newcommand{\pickclass}{\mathcal{P}}
\newcommand{\schurclass}{\mathcal{S}}
\newcommand{\sym}{\mathcal{H}(\R^{n})}
\newcommand{\vspan}{\operatorname{span}}
\newcommand{\image}{\operatorname{im}}
\newcommand{\kernel}{\operatorname{ker}}
\newcommand{\sesqui}{\mathcal{B}}
\newcommand{\unitary}{\mathcal{U}}
\newcommand{\normal}{\mathcal{N}}
\newcommand{\supp}{\text{supp}}
\newcommand{\rank}{\text{rank}}
\newcommand{\cone}{\text{cone}}
\newcommand{\dist}{\text{dist}}


\theoremstyle{plain}
\newtheorem{lause}[equation]{Theorem}
\newtheorem{lem}[equation]{Lemma}
\newtheorem{prop}[equation]{Proposition}
\newtheorem{kor}[equation]{Corollary}

\theoremstyle{definition}
\newtheorem{maar}[equation]{Definition}
\newtheorem{konj}[equation]{Conjecture}
\newtheorem{esim}[equation]{Example}
\newtheorem{phil}[equation]{Philosophy}
\newtheorem{quest}[equation]{Question}

%\theoremstyle{remark}
\newtheorem{huom}[equation]{Remark}

\pagestyle{plain}
\setcounter{page}{1}
\addtolength{\hoffset}{-1.15cm}
\addtolength{\textwidth}{2.3cm}
\addtolength{\voffset}{0.45cm}
\addtolength{\textheight}{-0.9cm}

\newcommand{\addchap}[1]{
    \newpage
    \input{chapters/#1}
}

