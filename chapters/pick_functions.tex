\chapter{Pick-Nevanlinna functions}

\textit{Pick-Nevanlinna function} is an analytic function defined in upper half-plane with a non-negative imaginary part. Such functions are sometimes also called Herglotz or $\R$ functions; we will call them just \textit{Pick functions}. The class of Pick functions is denoted by $\pickclass$.

\section{Examples and basic properties}

Most obvious examples of Pick functions might be functions of the form $\alpha z + \beta$ where $\alpha, \beta \in \R$ and $\alpha \geq 0$. Of course one could also take $\beta \in \Hpc$. As non-constant analytic functions are open mappings, real constants are the only Pick functions failing to map $\Hp \to \Hp$.

Pick functions can be thought of a set of ``positive analytic functions".

\begin{lause}
	$\pickclass \subset \{\text{analytic maps on $\Hp$} \}$ is a closed convex cone.
\end{lause}
\begin{proof}
	Again, this follows immediately from \ref{positive_machine}.
\end{proof}

Also a composition of Pick functions is a Pick function.

The map $z \mapsto -\frac{1}{z}$ is evidently a Pick function. Hence are also all functions of the form
\begin{align*}
	\alpha z + \beta + \sum_{i = 1}^{N} \frac{m_{i}}{\lambda_{i}- z},
\end{align*}
where $N$ is non-negative integer, $\alpha, m_{1}, m_{2}, \ldots, m_{N} \geq 0$, $\beta \in \Hp$ and $\lambda_{1}, \ldots, \lambda_{N} \in \Hm$.

There are (luckily) more interesting examples. All the functions of the form $x^{p}$ (with natural branch) where $0 < p < 1$ are Pick functions; similarly for $\log$. Another classic example is $\tan$. Indeed, by the addition formula
\begin{eqnarray*}
	\tan(x + i y) &=& \frac{\tan(x) + \tan(i y)}{1 - \tan(x) \tan(i y)} = \frac{\tan(x) + i \tanh(y)}{1 - i \tan(x) \tanh(y)} \\
	&=& \frac{\tan(x)(1 + \tanh^2(y))}{1 + \tan^2(x) \tanh^2(y)} + i \frac{(1 + \tan^2(x))\tanh(y)}{1 + \tan^2(x) \tanh^2(y)},
\end{eqnarray*}
and $y$ and $\tanh(y)$ have the same sign.

$\pickclass$ is almost salient: if $\varphi$ is analytic and $\Im(\varphi) = 0$, then $\varphi$ is a real constant (by Cauchy-Riemann equations, for instance). And again, this suggests that one should think about Pick functions up to a real constant.

So far we have made no mention on the topology, as it's usually taken to be the topology of locally uniform convergence. This definitely works (as it makes the evaluation functionals continuous), but we can do much better. It namely turns out that we can consider the set of Pick functions as a proper cone of $\C^{\Hp}$, set of all functions, with the topology of pointwise convergence.

\begin{prop}\label{pick_convergence}
	If $(\varphi_{i})_{i = 1}^{\infty}$ is a sequence of Pick functions converging pointwise, the limit function is also a Pick function.
\end{prop}

This result far from clear: pointwise limits of analytic functions need not analytic in general. We will not prove the result yet, but it strongly suggests that there is something more going on; Pick functions are very rigid. Note also that if Pick functions are thought of a subset of all functions, the definition of the cone doesn't really fit the general framework of theorem \ref{positive_machine}. This suggests that we are missing some better functionals, or better predual.

\section{Boundary}

To understand the rigidity phenomena we take a brief look at a close relative to Pick functions, \textit{Schur functions}. Schur funtions are analytic maps from open unit disc to closed unit disc. Classic fact about these functions is the Schwarz lemma.

\begin{lause}[Schwarz lemma]
	Let $\psi : \D \to \D$ be analytic such that $\psi(0) = 0$. Then $|\psi(z)| \leq |z|$ for any $z \in \D$.
\end{lause}

The textbook proof is based on two observations about analytic functions.
\begin{itemize}
	\item If $\varphi$ is analytic at $a$ with $\varphi(a) = 0$, then $\varphi/(\cdot - a)$ is also analytic.
	\item If $\varphi$ is analytic on closed unit disc and $|\varphi| \leq 1$ on the boundary of the disc, then $|\varphi| \leq 1$ inside the disc.
\end{itemize}

The first observation might not be very surprising, and it holds for smooth functions also. The second, on the other hand, is a true manifestation of the nature of the analytic maps: we can bound analytic functions simply by bounding them on the boundary of the domain. More generally: one knows everything about an analytic function on a domain simply by knowing it on a boundary, by Cauchy's integral formula.

This suggests that we should be able to recognize also Pick functions looking only at their boundary values. Actually even more is true: it suffices to look at the imaginary parts.

\begin{prop}
	Let $\varphi : U \to \C$ be analytic, such that $\overline{\Hp} \subset U$, and $\varphi$ is continuous at $\infty$. Then if the imaginary part of $\varphi$ is non-negative on the real axis, $\varphi$ is Pick function.
\end{prop}
\begin{proof}
	This follows immediately from the minimum principle applied to the harmonic function $\Im(\varphi)$.
\end{proof}

\section{Integral representations}

Recall that imaginary part of an analytic function determines also its real part, up to a constant, so we can also recover the function itself. This can be also done explicitly.

\begin{lause}
	Let $\varphi : U \to \C$ be analytic, such that $\overline{\Hp} \subset U$, and $\varphi(z) = O(|z|^{-\varepsilon})$ for some $\varepsilon > 0$ at infinity. Then for any $z \in \Hp$ we have
	\begin{align*}
		\varphi(z) = \frac{1}{\pi}\int_{\R} \frac{\Im(\varphi)(\lambda)}{\lambda - z} d \lambda
	\end{align*}
\end{lause}
\begin{proof}
	Note that the integral defines an analytic function, imaginary part of which equals
	\begin{align*}
		\frac{\Im(z)}{\pi}\int_{\R} \frac{\Im(\varphi)(\lambda)}{(\lambda - z)(\lambda - \overline{z})} d \lambda.
	\end{align*}
	This expression however equals $\Im(\varphi(z))$ by Poisson integral formula. By letting $z \to \infty$ one sees that also the real constants match.

	Alternatively one could observe that for a closed counter clockwise oriented curves $\gamma$ on the upper half-plane, enclosing $z$, we have
	\begin{align*}
		\varphi(z) = \frac{1}{2\pi i}\int_{\gamma} \frac{\varphi(\lambda)}{\lambda - z} d \lambda.
	\end{align*}
	Now given the bound, we may deform the contour to real axis. By comparing this identity and our goal, we are left to prove that
	\begin{align*}
		\frac{1}{2\pi i}\int_{\gamma} \frac{\overline{\varphi(\lambda)}}{\lambda - z} d \lambda = \frac{1}{2\pi i} \overline{\int_{\gamma} \frac{\varphi(\lambda)}{\lambda - \overline{z}} d \lambda}.
	\end{align*}
	But this is clear as $\varphi/(\cdot - \overline{z})$ is analytic in the upper half-plane.
\end{proof}

There's of course nothing really special about the decay assumption $\varphi(z) = O(|z|^{-\varepsilon})$; it's there just to make everything converge.

One can guarantee the convergence also by other means. Note that as the integrand behaves like $(\lambda - z)^{-1}$, if we subtract from it something (not depending on $z$) behaving the same way at the infinity, we ought to improve convergence, but only change the value of the function by a constant. As an example, consider the integral
\begin{align}\label{easy_pick_repr}
	\frac{1}{\pi}\int_{\R} \left(\frac{1}{\lambda - z} - \frac{[|\lambda| > 1]}{\lambda} \right) \Im(\varphi)(\lambda)d \lambda.
\end{align}
It converges to an analytic function as long as, say, $\Im(\varphi)$ is bounded. As before, its imaginary part coincides with $\varphi$'s so the functions are equal up to a real constant. Now, however, there's no reason for the real constants to match and indeed they need not.

Note that the previous idea could be used to construct Pick functions. Everything still makes sense if we replace $\Im(\varphi)$ by some other positive function, as long as the integral converges. Heck, we could replace it by any positive measure for which $\mu((\lambda^2 + 1)^{-1}) < \infty$.

(Almost) all the examples given before are actually just special cases of this construction. The rational functions $\frac{1}{\lambda - z}$, where $\lambda \in \R$ are obtained by setting $\mu = \delta_{\lambda}$. The power functions are obtained as
\begin{align*}
	z^{p} &= 1 + \frac{1}{\pi}\int_{-\infty}^{0} \left(\frac{1}{\lambda - z} - \frac{1}{\lambda - 1}\right) \Im(\lambda^{p}) d \lambda \\
	&=1 + \frac{1}{\pi}\int_{-\infty}^{0} \left(\frac{1}{\lambda - z} - \frac{1}{\lambda - 1}\right) |\lambda|^{p} \sin(\pi p) d \lambda,
\end{align*}
Logarithm is even simpler:
\begin{align*}
	\log(z) = \int_{-\infty}^{0} \left(\frac{1}{\lambda - z} - \frac{1}{\lambda - 1}\right) d \lambda.
\end{align*}
Tangent function could be obtained by putting $\delta$-measures to its poles, the points of the form $\frac{\pi}{2} + n \pi$, where $n \in \Z$.

The only exception is the function $z \mapsto \alpha z$ -- it can't be expressed as such integral. But even this failure is really more about poor point of view, as we will see in a minute. With these observations in mind it ought to be not too surprising that we have the following.

\begin{lause}\label{pick_nevanlinna_herglotz_representation_theorem}
	$\varphi \in \pickclass$, if and only
	\begin{align}\label{pick_representation}
		\varphi(z) = \alpha z + \beta + \int_{-\infty}^{\infty} \left(\frac{1}{\lambda - z} - \frac{\lambda}{\lambda^2 + 1}\right) d \mu(\lambda)
	\end{align}
	for some $\alpha \geq 0$ and $\beta \in \R$ and a Radon measure $\mu$ with $\int_{-\infty}^{\infty} (\lambda^2 + 1)^{-1} d \mu(\lambda) < \infty$.
\end{lause}

Choosing $\lambda \mapsto \frac{\lambda}{\lambda^2 + 1}$ is common choice in the literature and is convenient as
\begin{itemize}
	\item It's real, so the integrand is Pick function for any $\lambda \in \R$.
	\item We may recover the constant $\beta$ as $\Re(\varphi(i))$.
\end{itemize}

To better explain the appearance of the linear term, we can write the integral in a sligtly different form. Denoting $d \nu(\lambda) = \frac{d \mu(\lambda)}{\lambda^2 + 1}$, the formula reads
\begin{align*}
	\varphi(z) = \alpha z + \beta + \int_{-\infty}^{\infty} \frac{\lambda z + 1}{\lambda - z} d \nu(\lambda).
\end{align*}
Here $\nu$ is just a finite Borel measure. Now it kind of makes sense to extend the domain of this measure to infinity: the linear term merely corresponds to $\delta$-measure at infinity point. Of course, should one formalize this line of thought, the question on the type of extended real line had to be asked and one should address the topology. The answer is that one should glue the real line into a circle. One shouldn't worry about such issues, though, as these thoughts are here merely for intuition. The giveaway is that $\alpha$ should be really thought as a part of the measure $\mu$, even though this might not make perfect sense.

We will not prove theorem \ref{pick_nevanlinna_herglotz_representation_theorem}, but it shall work as a motivation. In order to understand Pick functions, we should understand their boundaries.

We will call the family

\begin{align*}
	\{z \mapsto z\} \cup \left\{z \mapsto \frac{1}{\lambda - z} | \lambda \in \R\right\}
\end{align*}

\textbf{extreme Pick functions}. Finite positive linear combinations of the extreme Pick functions are called \textbf{simple Pick functions}. Finally, we will call a Pick function \textbf{extentable}, if it is bounded and analytically extends over the real line. Such Pick functions enjoy representations of the form \ref{easy_pick_repr}.
\section{Pick functionals}

\begin{quest}
	How can we recover the measure $\mu$ from the Pick function?
\end{quest}

If $\mu$ is given by a continuous, bounded function, i.e. $d \mu(\lambda) = f(\lambda) d \lambda$, it's not very hard to see that
\begin{align*}
	f(\lambda) = \frac{1}{\pi}\lim_{y \to 0^{+}} \Im(\varphi(\lambda + i y)).
\end{align*}

This doesn't however work with rational functions with poles on the real line. One might try to salvage the situation by saying that poles should correspond to $\delta$-measures, but even if that would be true in some sense, we are only scratching the surface. What to do if the measure is for instance the uniform measure on the Cantor set?

Beauty of the measure theory is of course that we don't even need to make sense of the measure pointwise; everything is hidden in the averages. Could we recover the measures of open intervals then? Is it true that
\begin{align*}
	\mu((a, b)) = \lim_{y \to 0^{+}}\frac{1}{\pi}\int_{a}^{b} \Im(\varphi(\lambda + i y)) d \lambda?
\end{align*}
Even this isn't quite true: the problem is that if $\mu$ contains $\delta$-measure at $a$ or at $b$, the right-hand side doesn't see this properly. It turns out that this is only problem though.

We have however already encountered much better averages: the imaginary part of a Pick function $\varphi$ is a weighted average of the imaginary parts of $\varphi$ on the real line. We only proved this in the case of bounded $\varphi$, and indeed, the proper generalization should be 
\begin{align*}
	\Im(\varphi(z)) = \alpha \Im(z) + \frac{\Im(z)}{\pi}\int_{\R} \frac{d \mu(\lambda)}{(\lambda - z)(\lambda - \overline{z})}.
\end{align*}
Bear in mind that we also consider the constant $\alpha$ to be part of the measure. One can take this idea much further: if $q$ is any rational function with simple poles, no poles on $\R$, and decay $O(|z|^{-2})$ at infinity, the expression
\begin{align*}
	\int_{\R} q(\lambda) d \mu(\lambda)
\end{align*}
makes sense. Even better, partial fraction expansion allows us to write this integral in terms of as a linear combination of values of $\varphi$ and its conjugate. Indeed, we have
\begin{align*}
	q(\lambda) = c_{0}\frac{\lambda}{\lambda^2 + 1} + \sum_{i = 1}^{M} c_{i} \left(\frac{1}{\lambda - a_{i}} - \frac{\lambda}{\lambda^2 + 1}\right),
\end{align*}
where $a_{i}$'s are the poles of $q$. The decay condition tells that $c_{0} = 0$. If $\Im(a_{i}) > 0$, term corresponds to multiple of $\varphi(a_{i})$, and if $\Im(a_{i}) < 0$, we get the conjugate. Explicitly: if we abuse notation a tad by writing $\varphi(a_{i}) = \overline{\varphi(\overline{a_{i}})}$ if $\Im(a_{i}) < 0$, we have
\begin{align*}
	\int_{\R} q(\lambda) d \mu(\lambda) &= \int_{\R} \sum_{i = 1}^{M} c_{i} \left(\frac{1}{\lambda - a_{i}} - \frac{\lambda}{\lambda^2 + 1}\right)\mu(\lambda) \\
	&= \sum_{i = 1}^{M} c_{i} \left(\varphi(a_{i}) - \alpha a_{i} - \beta\right),
\end{align*}
which rewrites to
\begin{align*}
	\sum_{i = 1}^{M} c_{i} \varphi(a_{i}) &= \sum_{i = 1}^{M} c_{i} \left(\alpha a_{i} + \beta\right) + \int_{\R} q(\lambda) d \mu(\lambda).
\end{align*}
We can get some kind of glimpse of the measure just by looking at linear combinations of the values of Pick function.

These ideas get really useful when $q$ is non-negative on the real line. It follows easily from the lemma \ref{polynomial_lemma} that such rational functions can be written in the form
\begin{align*}
	q(\lambda) = \left( \sum_{i = 1}^{n} \frac{c_{i}}{\lambda - \lambda_{i}}\right) \left(\sum_{i = 1}^{n}\frac{\overline{c_{i}}}{\lambda - \overline{\lambda_{i}}}\right)
\end{align*}
where $\lambda_{1}, \lambda_{2}, \ldots, \lambda_{n} \in \C \setminus \R$ and $c_{1}, c_{2}, \ldots, c_{n} \in \C$. Running through the same calculations we see that

\begin{align*}
	\sum_{1 \leq i, j \leq n} c_{i} \overline{c_{j}} \frac{\varphi(\lambda_{i}) - \overline{(\varphi(\lambda_{j}))}}{\lambda_{i} - \overline{\lambda_{j}}} = \alpha \left|\sum_{i = 1}^{n} c_{i}\right|^2 + \int_{\R} q(\lambda) d \mu (\lambda) \geq 0.
\end{align*}

\begin{maar}
	Functional in $\left(\C^{\Hp}\right)^{*}$ of the form
	\begin{align*}
		\sum_{1 \leq i, j \leq n} c_{i} \overline{c_{j}} \frac{\delta_{\lambda_{i}} - \overline{\delta_{\lambda_{j}}}}{\lambda_{i} - \overline{\lambda_{j}}},
	\end{align*}
	$\lambda_{1}, \lambda_{2}, \ldots, \lambda_{n} \in \C \setminus \R$ and $c_{1}, c_{2}, \ldots, c_{n} \in \C$, is called a \textbf{Pick functional}. In other words
	\begin{align*}
		\varphi \mapsto \sum_{1 \leq i, j \leq n} c_{i} \overline{c_{j}} \frac{\varphi(\lambda_{i}) - \overline{\varphi(\lambda_{j})}}{\lambda_{i} - \overline{\lambda_{j}}}.
	\end{align*}
\end{maar}

Given $\lambda_{1}, \lambda_{2}, \ldots, \lambda_{n} \in \C \setminus \R$, the matrix
\begin{align*}\label{Pick_matrix}
	\begin{bmatrix}
		[\lambda_{1}, \overline{\lambda_{1}}]_{\varphi} & [\lambda_{1}, \overline{\lambda_{2}}]_{\varphi} & \cdots & [\lambda_{1}, \overline{\lambda_{n}}]_{\varphi} \\
		[\lambda_{2}, \overline{\lambda_{1}}]_{\varphi} & [\lambda_{2}, \overline{\lambda_{2}}]_{\varphi} & \cdots & [\lambda_{2}, \overline{\lambda_{n}}]_{\varphi} \\
		\vdots & \vdots & \ddots & \vdots \\
		[\lambda_{n}, \overline{\lambda_{1}}]_{\varphi} & [\lambda_{n}, \overline{\lambda_{2}}]_{\varphi} & \cdots &  [\lambda_{n}, \overline{\lambda_{n}}]_{\varphi}
	\end{bmatrix}
\end{align*}
is called a \textbf{Pick matrix}. Pick functionals are simply values of quadratic forms of Pick matrices.

We denote the set of Pick functionals by $\pickclass^{*}$.

\begin{lause}\label{pick_functionals}
	Let $p^{*} \in \left(\C^{\Hp}\right)^{*}$. Then the following are equivalent.
	\begin{enumerate}[(i)]
		\item $p^{*} \in \pickclass^{*}$
		\item $p^{*}(\varphi) \geq 0$ for any extreme Pick function $\varphi$.
		\item $p^{*}(\varphi) \geq 0$ for any extentable Pick function $\varphi$.
		\item $p^{*}(\varphi) \geq 0$ for any Pick function $\varphi$.
	\end{enumerate}
\end{lause}
\begin{proof}
	The implications $(i) \Rightarrow (ii) \Rightarrow (iii)$ and $(iv) \Rightarrow (ii)$ are clear, up to one detail: if $p^{*}(\varphi) \geq 0$ for any extreme Pick function $\varphi$, we should prove that $p^{*}(1) = 0$. But this follows as soon as one notes that
	\begin{align*}
		p^{*}\left(\frac{1}{\lambda - \cdot}\right) = \frac{p^{*}\left(1\right)}{\lambda} + O\left(\frac{1}{\lambda^2}\right).
	\end{align*}

	$(iii) \Rightarrow (iv)$: It suffices to prove that the extentable Pick functions are dense (with respect to the topology of pointwise convergence) in the set of all Pick function. But for this it is enough to find a sequence of Pick functions $(g_{n})_{n = 1}^{\infty}$ such that
	\begin{enumerate}
		\item $g_{n}(z) \to z$ pointwise as $n \to \infty$,
		\item $g_{n}$'s extend analytically over real line and $g_{n}(\overline{\H_{+}})$ is compact subset of $\H_{+}$ for every $n \geq 1$,
	\end{enumerate}
	as then we have $\varphi \circ g_{n} \to \varphi$ pointwise as $n \to \infty$ for every Pick function $\varphi$ the functions $\varphi \circ g_{n}$ are evidently bounded and extend analytically over the real line.

	It is not very hard to check that we may take
	\begin{align*}
		g_{n}(z) = \frac{z + \frac{i}{n}}{1 - \frac{i z}{n}}.
	\end{align*}

	$(ii) \rightarrow (i)$: By the construction of the $\pickclass^{*}$, it contains all the functionals with finite support, which give positive values for all extreme Pick functions. Thus it remains to be noted that all the continuous functionals in $\left(\C^{\Hp}\right)^{*}$ have finite support. But this follows from the general fact that the dual of a product equals direct sum of the duals.
\end{proof}

It is useful to note that one does not even need to test every extreme Pick function to check that functional is Pick functional, dense subset suffices. This is clear for the functions $z \mapsto \frac{1}{\lambda - z}$ but also holds for $z \mapsto z$, when this is intepreted as the function with $\lambda = \infty$ (and dense subset refers to the circle topology). Indeed, we have
\begin{align*}
	p^{*}\left(\frac{1}{\lambda - \cdot}\right) = p^{*}\left(\frac{1}{\lambda - \cdot} - \frac{1}{\lambda}\right) = \frac{p^{*}(\cdot)}{\lambda^2} + O\left(\frac{1}{\lambda^3}\right).
\end{align*}

Finally, the following correspondence should be clear by now.

\begin{prop}
	There is a natural bijective ($\R$-)linear map between non-negative rational functions vanishing at infinity and $\pickclass^{*}$.
\end{prop}

This line of thought induces a dual pairing between rational functions $r$ with simple poles such that $(1 + (\cdot)^2) r$ is bounded on $\R$, and $\C^{\Hp}$. We denote this pairing by $\langle r, \varphi \rangle_{\pickclass}$ and the class of rational functions by $R_{\infty}(\Hp)$. For $X \subset \Hp$ we also denote by $R_{\infty}(X)$ those functions on $R_{\infty}(\Hp)$, whose poles are on $X$ (and on its reflection with respect to $\R$).

\section{Weakly Pick functions}

Theorem \ref{pick_functionals} implies that $\pickclass^{*}$ is really just the dual cone of $\pickclass$. It turns out that the ``converse" is also true: $\pickclass^{*}$ is also a predual of $\pickclass$.

\begin{maar}
	We will elements of $\left(\pickclass^{*}\right)^{*}$ \textbf{weakly Pick functions}.
\end{maar}

In layman's terms, weakly Pick functions are ones, which look like Pick functions if one only considers linear functionals. The aim of this section is to show that weakly Pick functions are exactly the Pick functions. We already proved one direction in the theorem \ref{pick_functionals}. 

The other direction is tricky. Note that weakly Pick maps map to the upper half-plane so the interesting part is to prove that weakly Pick maps are analytic. For this we are going to verify bounds for the divided differences of $\varphi$. Recall that by theorem \ref{bounded_div} it suffices to verify that the order $2$ divided differences are locally bounded to prove that $\varphi$ is (continuously) differentiable. Strictly speaking we only proved the result on real line, but the prove would be almost identical in the complex case.

The idea is the following: we are going to formulate everything terms of the linear functionals. This idea is best illustrated with an example.

\begin{lem}[Harnack inequality]\label{pick_harnack_lemma}
	Let $\varphi$ be a weakly Pick function. Then for every compact $K \subset \Hp$ there exists a constant $C_{K}$ such that
	\begin{align*}
		\frac{\Im(\varphi(z))}{\Im(z)} \leq C_{K} \frac{\Im(\varphi(w))}{\Im(w)}
	\end{align*}
	for every $z, w \in K$.
\end{lem}
\begin{proof}
	Note that the sought inequality can be rephrased as positivity of the linear functional
	\begin{align*}
		\varphi \mapsto C_{K} \frac{\Im(\varphi(w))}{\Im(w)} - \frac{\Im(\varphi(z))}{\Im(z)}.
	\end{align*}
	By theorem \ref{pick_functionals} it suffices to prove that there exists constant $C_{K}$ such that the previous inequality holds for any extreme Pick function. This implies that we should have
	\begin{align*}
		\frac{1}{|\lambda - z|^2} \leq \frac{C_{K}}{|\lambda - w|^2}
	\end{align*}
	for every $\lambda \in \R$. But
	\begin{align*}
		\left|\frac{\lambda - w}{\lambda - z}\right|^2 \leq \left|1 + \frac{z - w}{\lambda - z}\right|^2 \leq \left(1 + \frac{|z - w|}{\Im(z)}\right)^2,
	\end{align*}
	so we can definitely find such constant.
\end{proof}

Similarly, one can prove that weakly Pick functions are continuous.

\begin{lause}\label{pick_continuity_lemma}
	Let $\varphi$ be a weakly Pick function. Then $\varphi$ is continuous.
\end{lause}
\begin{proof}
	Our aim is to bound the divided difference $|[z, w]_{\varphi}|$. Now the problem is that this expression is not linear in the function anymore. There's a way to fix this problem however: we bound $\Re(\omega [z, w]_{\varphi})$ for $\omega \in \unitcircle$. This expression is linear in the function, and we have
	\begin{align*}
		|z| \leq C \Leftrightarrow \Re(\omega z) \leq C \text{ for every $\omega \in \unitcircle$}.
	\end{align*}
	Observe that
	\begin{align*}
		\Re\left(\frac{\omega}{(\lambda - z)(\lambda - w)} \right) \leq \frac{1}{|\lambda - z||\lambda - w|} \leq \frac{1}{2} \left( \frac{1}{|\lambda - z|^2} + \frac{1}{|\lambda - w|^2}\right)
	\end{align*}
	for every $\omega \in \unitcircle$. It follows that
	\begin{align*}
		|[z, w]_{\varphi}| \leq \frac{1}{2} \left(\frac{\Im(\varphi(z))}{\Im(z)} + \frac{\Im(\varphi(w))}{\Im(w)}\right)
	\end{align*}
	for any weakly Pick function. Combining this with Harnack inequality \ref{pick_harnack_lemma} yields that any weakly Pick function is locally Lipschitz, so in particular continuous.
\end{proof}

The previous argument can be easily extended to the following.

\begin{lause}\label{Hindmarsh_theorem}
	Let $\varphi$ be a weakly Pick function. Then for any $n \geq 1$ and $z_{0}, z_{1}, \ldots, z_{n}$ we have
	\begin{align*}
		|[z_{0}, z_{1}, \ldots, z_{n}]_{\varphi}| \leq \frac{1}{\Im(z_{2}) \Im(z_{3}) \ldots \Im(z_{n})} \frac{1}{2}\left(\frac{\Im(\varphi(z_{0}))}{\Im(z_{0})} + \frac{\Im(\varphi(z_{1}))}{\Im(z_{1})}\right).
	\end{align*}
	In particular any weakly Pick function is analytic and hence a Pick function.
\end{lause}
\begin{proof}
	Simply note that
	\begin{align*}
		\Re\left(\frac{\omega}{(\lambda - z_{0})(\lambda - z_{1}) \cdots (\lambda - z_{n})} \right) \leq \frac{1}{\Im(z_{2}) \Im(z_{3}) \ldots \Im(z_{n})} \frac{1}{|\lambda - z_{0}||\lambda - z_{1}|}
	\end{align*}
	and follow the argument in the proof of theorem \ref{pick_continuity_lemma}.
\end{proof}

It is worthwhile to note that as one really only needs to get bound for order $2$ divided differences in the proof of \ref{Hindmarsh_theorem}, one only needs to keep track of small family of pick functionals, in particular ones with support of at most $3$ points. This observation is known as the Hindmarsh theorem.

\begin{kor}\label{pick_weakly_pick}
	$\varphi : \Hp \to \C$ is weakly Pick, if and only if it is Pick function.
\end{kor}

\begin{proof}[Proof of theorem \ref{pick_convergence}]
	This follows immediately from \ref{pick_weakly_pick}.
\end{proof}

\section{Pick-Nevanlinna extension theorem}

There's a remarkable generalization to the theorem \ref{pick_weakly_pick}.

\begin{maar}
	Let $X \subset \Hp$. We say that $\varphi : X \to \C$ is weakly Pick on $X$ if $p^{*}(\varphi) \geq 0$ for any Pick functional $p^{*}$ supported on $X$.
\end{maar}

\begin{lause}[Pick-Nevanlinna extension theorem]\label{open_pick_interpolation}
	Let $U \subset \Hp$ be open and assume that $\varphi$ is weakly Pick on $U$. Then there exists a unique pick function $\tilde{\varphi}$ such that $\restr{\tilde{\varphi}}{U} = \varphi$.
\end{lause}

The proof of this result is based on the following observation:

\begin{lem}\label{open_pick_lemma}
	Let $U \subset \Hp$ be open and assume that $\varphi$ is weakly Pick on $U$. Let $z_{0} \in U$. Then there exists unique weakly Pick $\tilde{\varphi} : U \cap \D(z_{0}, \Im(z_{0}))$ such that $\restr{\tilde{\varphi}}{U} = \varphi$.
\end{lem}

\begin{proof}
	Take any sequence $z_{1}, z_{2}, \ldots \in U$ converging to $z_{0}$: we claim the Newton series with nodes $z_{1}, z_{2}, \ldots$ gives the (by analyticity necessarily unique) extension for $\varphi$ to $\D(z_{0}, \Im(z_{0})) \setminus U$.

	To this end take any Pick functional $p^{*}$ supported on  $\D(z_{0}, \Im(z_{0})) \cup U$ and apply it to our $\tilde{\varphi}$. The functional correponds to some $r \in R_{\infty}(\D(z_{0}, \Im(z_{0})) \cup U)$. Now if we replace all the evaluations of $p^{*}$ at $\D(z_{0}, \Im(z_{0})) \setminus U$ by truncation of Newton series (with $N$ terms), we can interpret the result as a new linear functional, say $p^{*}_{N}$. The corresponding rational function is also changed (say to $r_{N} \in R_{\infty}(U)$): all the terms of the form $(\lambda - w_{0})^{-1}$ for $w_{0} \in \D(z_{0}, \Im(z_{0})) \setminus U$ are replaced by
	\begin{align*}
		\frac{1}{\lambda - z_{1}} + \frac{(w_{0} - z_{1})}{(\lambda - z_{1}) (\lambda - z_{2})} + \ldots + \frac{(w_{0} - z_{1})\cdots (w_{0} - z_{N - 1})}{(\lambda - z_{1})\cdots (\lambda - z_{N})},
	\end{align*}
	and similarly for conjugate terms. Difference between these rational functions can be easily bounded by
	\begin{align*}
		\left(\frac{|w_{0} - z_{0}|}{\Im(z_{0})}\right)^{N}\frac{C}{|\lambda - z_{0}|^2},
	\end{align*}
	where $C$ is some constant not depending on $N$. But this means that $r_{N}$ can't be too small, as $r$ was non-negative to begin with. Indeed, by summing over all the evaluations of $p^{*}$ at $\D(z_{0}, \Im(z_{0})) \setminus U$, we see that
	\begin{align*}
		r_{N} \geq -\frac{C'}{|\lambda - z_{0}|^2} \rho^{N}
	\end{align*}
	for some $\rho < 1$ and $C' > 0$ (again, not depending on $N$). It follows that
	\begin{align*}
		p^{*}(\tilde{\varphi}) = \lim_{N \to \infty} p^{*}_{N}(\varphi) \geq \lim_{N \to \infty} -C' \frac{\Im(\varphi(z_{0}))}{\Im(z_{0})} \rho^{N} = 0,
	\end{align*}
	hence the claim.
\end{proof}


\begin{proof}[Proof of theorem \ref{open_pick_interpolation}]
	Consider all weakly Pick extensions of $\varphi$ (to open supersets of $U$), ordered by restriction. These maps trivially satisfy conditions of Zorn's lemma so we may Pick maximal map, $\tilde{\varphi}$. It follows immediately from lemma \ref{open_pick_lemma} that the domain of $\tilde{\varphi}$ is the whole $\Hp$.

	Of course, Zorn's lemma is not really necessary here: one could write explicit extension scheme (TODO: picture).
\end{proof}

\section{Pick-Nevanlinna interpolation theorem}

Although Pick-Nevanlinna extension theorem is strong enough tool for our purposes, one cannot simply talk about it without discussing also its big brother, interpolation theorem.

\begin{lause}[Pick-Nevanlinna interpolation theorem]\label{pick_interpolation}
	Let $X \subset \Hp$ be arbitrary and assume that $\varphi$ is weakly Pick on $X$. Then there exists pick function $\tilde{\varphi}$ such that $\restr{\tilde{\varphi}}{U} = \varphi$.
\end{lause}

It's easy to see that such extension it not unique in general.

The proof is based on the following result.

\begin{lem}\label{pick_extension_lemma}
	Let $X \subset \Hp$ non-empty and $z_{0} \in \Hp \setminus X$. Assume that $\varphi$ is weakly Pick on $X$. Then $\varphi$ can be extended to $z_{0}$, in such a way that the extension is weakly Pick $X \cup \{z_{0}\}$. Moreover, the set of possible values of the extension at $z_{0}$ is a compact subset of $\Hp$.
\end{lem}

Let us first proof the theorem given this lemma.

\begin{proof}[Proof of theorem \ref{pick_interpolation}]
	Consider family of all weakly Pick extensions of $\varphi$ ordered by restriction. It is clear that this family satisfies the conditions of the Zorn's lemma and hence it has a maximal element. But by the previous lemma domain of this maximal element has to have the whole $\Hp$, so it is a sought extension.
\end{proof}

\begin{proof}[Proof of lemma \ref{pick_extension_lemma}]
	Let us first deal with the case of finite $X$. The idea is somewhat similar to the proof of \ref{cheapSpectral}: while in general the extension is very much not unique, if the situation is restricted enough, we are in better situation. Let us denote the sought extension by $\tilde{\varphi}$. Assume first that $\varphi$ is degenerate in $X$ in the sense that $\langle \varphi, r\rangle_{\pickclass} = 0$ for some non-zero $r \in R_{\infty}(X)$. We may clearly assume that $r$ has only real roots. Indeed, if $r(z_{1}) = 0$ for some $\Im(z_{1}) > 0$, let $\tilde{r} := r \Im(z_{1})^2/N(\cdot - z_{1}) \in R_{\infty}(X)$. As $\tilde{r} \leq r$, also  $\langle \varphi, \tilde{r}\rangle_{\pickclass} = 0$, and $\tilde{r}$ has less non-real roots than $r$.

	Note that since $(\lambda - z_{0})^{-1}$ is bounded on $\R$, by say $M$, we should have
	\begin{align*}
		\left|\langle r \frac{1}{\cdot - z_{0}}, \tilde{\varphi} \rangle_{\pickclass}\right| \leq M \langle r , \varphi \rangle_{\pickclass} = 0,
	\end{align*}
	and hence $\langle r (\cdot - z_{0})^{-1}, \tilde{\varphi} \rangle_{\pickclass} = 0$. Note that then also $\langle r (\cdot - \overline{z_{0}})^{-1}, \tilde{\varphi} \rangle_{\pickclass} = 0$. As $r(z_{0}) \neq 0$, this determines the value $\tilde{\varphi}(z_{0})$.

	Now it remains to be proven that with this choice $\tilde{\varphi}$ is weakly Pick. To this end take any $s \in R_{\infty}(X \cup \{z_{0}\})$. Note that for any $a \in \C$ and $\lambda \in \R$ we have
	\begin{align*}
		\tilde{s}(\lambda) := s(\lambda) + \left(\frac{a}{\lambda - z_{0}} + \frac{\overline{a}}{\lambda - \overline{z_{0}}}\right) r(\lambda) + 2 |a| M r(\lambda) \geq 0.
	\end{align*}
	By picking $a$ suitably, $\tilde{s}$ doesn't have pole at $z_{0}$ (or $\overline{z_{0}}$), and since $\varphi$ is weakly Pick, we thus have
	\begin{align*}
		\langle s, \tilde{\varphi} \rangle_{\pickclass} = \langle \tilde{s}, \varphi \rangle_{\pickclass} \geq 0,
	\end{align*}
	the claim.

	If $\varphi$ is not-degenerate, we may certainly find $c_{0} > 0$ such that $\varphi_{c_{0}} := \varphi - c_{0} i$ is weakly Pick on $X$ and degenerate, and if we find extension for $\varphi_{c_{0}}$, we get one also for $\varphi$.

	The proof of \ref{pick_continuity_lemma} immediately implies that the set of suitable values $\tilde{\varphi}(z_{0})$ bounded, and as it is also clearly closed, it is compact.

	Let us now move to the case of general non-empty $X$. For any finite subset $F \subset X$ denote the set of possible values of a weakly Pick extension of $\restr{\varphi}{F}$ at $z_{0}$ by $W_{F}$. We clearly have $W_{F_{1} \cup F_{2}} \subset W_{F_{1}} \cap W_{F_{2}}$ and hence
	\begin{align*}
		\{ W_{F} | \text{$F$ is a finite subset of $X$}\}
	\end{align*}
	is a family of compact sets with finite intersection property. Consequently their intersection is non-empty and compact.
\end{proof}

Again, one could avoid the use of Zorn's lemma by, for instance, first extending $\varphi$ to dense subset of $\Hp$ and then noting that this extension continuously extends to the whole of $\Hp$, to a weakly Pick function.


\section{Notes and references}

\begin{comment}

TODO:
\begin{itemize}
	\item Examples of representing measures behind functions and functions behind representing measures
	\item Spectral commutant lifting theorem
	\item Use Morera's theorem to prove weak Hindmarsh's theorem
	\item Nice formula for finite Pick extension (rational function case)
\end{itemize}

\end{comment}






