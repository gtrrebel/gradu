\chapter{Pick-Nevanlinna functions}

\textit{Pick-Nevanlinna function} is an analytic function defined in upper half-plane with a non-negative real part. Such functions are sometimes also called Herglotz or $\R$ functions but we will call them just \textit{Pick functions}. The class of Pick functions is denoted by $\pickclass$.

Pick functions have many interesting properties related to positive matrices and that is why they are central objects to the theory of matrix monotone functions.

\section{Basic properties and examples}

Most obvious examples of Pick functions might be functions of the form $\alpha z + \beta$ where $\alpha, \beta \in \R$ and $\alpha \geq 0$. Of course one could also take $\beta \in \Hpc$. Actually real constants are the only Pick functions failing to map $\Hp \to \Hp$: non-constant analytic functions are open mappings.

Sum of two Pick functions is a Pick function and one can multiply a Pick function by non-negative constant to get a new Pick function. Same is true for composition.

The map $z \mapsto -\frac{1}{z}$ is evidently a Pick function. Hence are also all functions of the form
\[
	\alpha z + \beta + \sum_{i = 1}^{N} \frac{m_{i}}{\lambda_{i}- z},
\]
where $N$ is non-negative integer, $\alpha, m_{1}, m_{2}, \ldots, m_{N} \geq 0$, $\beta \in \Hp$ and $\lambda_{1}, \ldots, \lambda_{N} \in \Hp$. So far we have constructed our function by adding simple poles to the closure of lower half-plane. We could further add poles of higher order at lower half plane, and change residues and so on, but then we have to be a bit more careful.

There are (luckily) more interesting examples: all the functions of the form $x^{p}$ where $0 < p < 1$ are Pick functions. To be precise, one should choose branch for the previous so that they are real at positive real axis. Also $\log$ yields Pick function when branch is chosen properly i.e. naturally again. Another classic example is $\tan$. Indeed, by the addition formula
\begin{eqnarray*}
	\tan(x + i y) &=& \frac{\tan(x) + \tan(i y)}{1 - \tan(x) \tan(i y)} = \frac{\tan(x) + i \tanh(y)}{1 - i \tan(x) \tanh(y)} \\
	&=& \frac{\tan(x)(1 + \tanh^2(y))}{1 + \tan^2(x) \tanh^2(y)} + i \frac{(1 + \tan^2(x))\tanh(y)}{1 + \tan^2(x) \tanh^2(y)},
\end{eqnarray*}
and $y$ and $\tanh(y)$ have the same sign.

We observe the following useful fact.

\begin{prop}
	If $(\varphi_{i})_{i = 1}^{\infty}$ is a sequence of Pick functions converging locally uniformly, the limit function is also a Pick function.
\end{prop}
\begin{proof}
	Locally uniform limits of analytic functions are analytic. Also the limit function has evidently non-negative imaginary part.
\end{proof}

This is one of the main reasons we include real constants to Pick functions, although they are exceptional in many ways. Note that for any $z \in \Hp$ we have $\log(z) = \lim_{p \to 0^{+}}(z^p - 1)/p$: $\log$ can be understood as a limit of Pick functions.

There's a considerable strengthening of the previous result.

\begin{prop}\label{pick_convergence}
	If $(\varphi_{i})_{i = 1}^{\infty}$ is a sequence of Pick functions converging pointwise, the limit function is also a Pick function.
\end{prop}

We will not prove this quite surprising result yet; the message is that the class of Pick functions is very rigid in some sense.

\section{Rigidity}

\subsection{Schur functions}

To understand the rigidity phenomena we take look at the close relative to Pick functions, \textit{Schur functions}. Schur funtions are analytic maps from open unit disc to closed unit disc. These functions functions include for instance power functions $z \mapsto z^{n}$ and more generally, as one may check, any Blaschke products, that is products of the terms of the form
\[
	\rho_{a, \omega}(z) = \omega \frac{a - z}{1 - \overline{a} z},
\]
Blaschke factors. Classic fact about these functions is the Schwarz lemma.

\begin{lause}[Schwarz lemma]
	Let $\psi : \D \to \D$ be analytic such that $\psi(0) = 0$. Then $|\psi(z)| \leq |z|$ for any $z \in \D$ and hence also $|\psi'(0)| \leq 1$. If $|\psi(z)| = |z|$ for some $z \in \D \setminus \{0\}$ or $|\psi'(0)| = 1$, $\psi(z) = \omega z$ for some $\omega \in \mathbb{S}$.
\end{lause}

Fixing value of Schur function at $0$ restricts function a whole lot.

The usual proof is by cleverly using maximum modulus principle for $\psi(z)/z$. Maximum modulus principle itself is consequence of the Cauchy's integral formula. There's also more symmetric form for Schwarz lemma, called Schwarz-Pick theorem.

\begin{lause}[Schwarz-Pick theorem]
	Let $\psi : \D \to \D$ be analytic. Then for any $z_{1}, z_{2} \in \D$ we have
	\[
		\left|\frac{\psi(z_{1}) - \psi(z_{2})}{1 - \overline{\psi(z_{1})} \psi(z_{2})} \right| \leq \left|\frac{z_{1} - z_{2}}{1 - \overline{z_{1}}z_{2}}\right|,
	\]
	and
	\[
		\frac{\left|\psi'(z_{1})\right|}{1 - |\psi(z_{1})|^2} \leq \frac{1}{1 - |z_{1}|^2}.
	\]
	If the equality holds in one of the inequalities, $\psi$ is an Blaschke factor.
\end{lause}

Note that one obtains the usual Schwarz lemma if $z_{1} = 0 = \psi(z_{1})$. One may check that if $\psi$ is Blaschke factor, the inequalities hold as equalities.
\begin{proof}
	Consider the map $\psi_{1} = \rho_{\psi(z_{1})} \circ \psi \circ \rho_{z_{1}}$. The claim follows by using the previous form of the Schwarz lemma for the $\psi_{1}$ and point $z_{2}$.
\end{proof}

The previous result shows that Blaschke factors are exactly the analytic bijections $\D \to \D$.

One can also make weaker estimates straight from Cauchy's integral formula. We can for instance write
\[
	\psi(a) = \frac{1}{2 \pi i}\int_{\gamma} \frac{\psi(z)}{z - a} dz
\]
for suitable $\gamma$. Now if $\psi$ extends over unit circle, letting $\gamma$ trace unit circle we have
\[
	|\psi(a)| \leq \left(\max_{|z| = 1} \left|\psi(z)\right| \right) \frac{1}{2 \pi}\int_{0}^{2 \pi} \frac{d t}{|e^{i t}  - a|}.
\]
This is not very strong estimate; the averaged integral tends to infinity as $|a| \to 1$, but the point is that by a simple argument we have can some bound on analytic function on a disc given only bound on its boundary values.

The point in looking at Schur functions is that we can directly bring the claims for Schur functions to Pick functions with maps
\begin{align*}
	\toplane : \D \to \Hp & \hspace{1cm} \toplane (z) = i \frac{1 - z}{1 + z} \\
	\tocircle : \Hp \to \D & \hspace{1cm} \tocircle (z) = \frac{i - z}{i + z}.
\end{align*}
If $\psi$ is Schur function then $\toplane \circ \psi \circ \tocircle$ is a Pick function, and conversely every Pick function $\varphi$ gives rise to Schur function $\tocircle \circ \varphi \circ \toplane$. We can directly translate Schwarz-Pick theorem to Pick functions.

\begin{lause}[Schwarz-Pick theorem for the upper half-plane]
	Let $\varphi : \Hp \to \Hp$ be analytic. Then for any $z_{1}, z_{2} \in \Hp$ we have
	\[
		\left|\frac{\varphi(z_{1}) - \varphi(z_{2})}{\varphi(z_{1}) - \overline{\varphi(z_{2})}} \right| \leq \left|\frac{z_{1} - z_{2}}{z_{1} - \overline{z_{2}}} \right|.
	\]
\end{lause}

Correspondingly analytic bijections in $\pickclass$ are exactly function of the form
\[
	z \mapsto \frac{a z + b}{c z + d},
\]
where $a, b, c, d \in \R$ and $a d - b c > 0$. Among these mapping the map
\[
	f : z \mapsto \frac{z \Re(w_{0}) - (|w_{0}|^2 + \Im(w_{0}))}{(1 + \Im(w_{0})) z - \Re(w_{0}))}
\]
satisfies $f(i) = w_{0}$.

\subsection{Pick matrices}

We can also arrive at the previous estimate directly using Cauchy's integral theorem, a bit similarly as with unit disc. Assume first that $\varphi$ is a bounded Pick-function extending over real line. We can start by proving that if $\varphi$ has positive imaginary part on real line, then it has positive real part on whole upper half-plane The idea to consider
\begin{align*}
	\varphi(a) = \frac{1}{2 \pi i}\int_{\gamma} \frac{\varphi(z)}{z - a} dz,
\end{align*}
again for suitable closed curve $\gamma$. If $\varphi$ decays fast enough at infinitity, we can deform the contour $\gamma$ to coincide with real axis. The unfortunate thing is that the we can't say much about the real or imaginary part of $ \frac{\varphi(z)}{z - a}$. But there's a trick: we can fix our problem by considering
\begin{align*}
	\frac{1}{2 \pi i}\int_{-\infty}^{\infty} \frac{\varphi(z)}{(z - a) (z - \overline{a})} dz.
\end{align*}
This expression has positive real part, and by using the residue theorem, the real part equals
\[
	\frac{\varphi(a) - \overline{\varphi(a)}}{a - \overline{a}},
\]
hence the claim. Also boundedness is enough for this estimate.

To arrive at Schwarz-Pick theorem consider integral
\begin{align*}
	\frac{1}{2 \pi i}\int_{-\infty}^{\infty} \varphi(z) \left(\frac{c_{1}}{z - z_{1}} + \frac{c_{2}}{z - z_{2}} \right) \left(\frac{\overline{c_{1}}}{z - \overline{z_{1}}} + \frac{\overline{c_{2}}}{z - \overline{z_{2}}} \right) dz.
\end{align*}
Again, the point is that writing $h(z) = \frac{c_{1}}{z - z_{1}} + \frac{c_{2}}{z - z_{2}}$, $h$ is meromorphic in upper half-plane expect for the simple poles at $z_{1}$ and $z_{2}$, thus giving information about $\varphi(z_{1})$ and $\varphi(z_{2})$, $\overline{h(\overline{z})}$ is analytic in the upper halfplane and $h(z)\overline{h(\overline{z})}$ is real on the real axis.

Now this expression should have positive real part for any $c_{1}, c_{2} \in \C$, and computing the real part using the residue theorem, we arrive at
\begin{align*}
	[z_{1}, \overline{z_{1}}]_{\varphi} |c_{1}|^{2} &+ [z_{1}, \overline{z_{2}}]_{\varphi} c_{1} \overline{c_{2}} \\
	[z_{2}, \overline{z_{1}}]_{\varphi} c_{2} \overline{c_{1}} &+ [z_{2}, \overline{z_{2}}]_{\varphi} |c_{2}|^{2},
\end{align*}
where we abuse the notation a little by writing $\varphi(\overline{z}) = \overline{\varphi(z)}$.

But this is to say that the matrix
\begin{align*}
	\left([z_{i}, \overline{z_{j}}]_{\varphi}\right)_{1 \leq i, j \leq 2} =
	\begin{bmatrix}
		[z_{1}, \overline{z_{1}}]_{\varphi} & [z_{1}, \overline{z_{2}}]_{\varphi} \\
		[z_{2}, \overline{z_{1}}]_{\varphi} & [z_{2}, \overline{z_{2}}]_{\varphi}
	\end{bmatrix}
\end{align*}
is positive. Now, it just so turns out that ``determinant of the matrix is non-negative" is equivalent to Schwarz-Pick theorem.

It is not very hard to generalize the previous argument larger matrices, and we have arrived to
\begin{lause}\label{pick-nevanlinna_theorem}
	If $\varphi$ is Pick function and $z_{1}, z_{2}, \ldots, z_{n}$ are any points in the upper half-plane, then the matrix
	\begin{align}\label{Pick_matrix}
	\begin{bmatrix}
		[z_{1}, \overline{z_{1}}]_{\varphi} & [z_{1}, \overline{z_{2}}]_{\varphi} & \cdots & [z_{1}, \overline{z_{n}}]_{\varphi} \\
		[z_{2}, \overline{z_{1}}]_{\varphi} & [z_{2}, \overline{z_{2}}]_{\varphi} & \cdots & [z_{2}, \overline{z_{n}}]_{\varphi} \\
		\vdots & \vdots & \ddots & \vdots \\
		[z_{n}, \overline{z_{1}}]_{\varphi} & [z_{n}, \overline{z_{2}}]_{\varphi} & \cdots &  [z_{n}, \overline{z_{n}}]_{\varphi}
	\end{bmatrix}
	\end{align}
	is positive.
\end{lause}

The matrix \ref{Pick_matrix} is called \textit{Pick matrix}.

\begin{proof}
	We have already proved the theorem in the case that $\varphi$ is bounded and extends analytically over the real line. For general case consider the sequence $g_{n}$ of Pick functions given by
	\[
		g_{n}(z) = \frac{(1 - \frac{1}{n})x + \frac{i}{n}}{(1 - \frac{1}{n}) - i\frac{x}{n}} = \toplane \circ \left(z \mapsto \left(1 - \frac{2}{n}\right)z\right) \circ \tocircle.
	\]
	Now
	\begin{enumerate}
		\item $g_{n}(z) \to z$ pointwise.
		\item $g_{n}$'s extend analytically over real line and $g_{n}(\overline{\H_{+}})$ is compact subset of $\H_{+}$ for every $n \geq 1$.
	\end{enumerate}
	It follows $\varphi \circ g_{n} \to \varphi$ pointwise and $\varphi \circ g_{n}$'s satisfy the already proven case. Finally, also the corresponding Pick-matrices of $\varphi \circ g_{n}$'s converge to Pick matrix of $\varphi$, hence the general case.
\end{proof}

\section{Weak characterization}

Theorem \ref{pick-nevanlinna_theorem} has a converse.

\begin{lause}\label{pick-nevanlinna_converse}
	If $\varphi : \Hp \to \Hpc$ such that for any $n \geq 1$ and $z_{1}, z_{2}, \ldots, z_{n} \in \Hp$ the respective Pick matrix
	\begin{align*}
	\begin{bmatrix}
		[z_{1}, \overline{z_{1}}]_{\varphi} & [z_{1}, \overline{z_{2}}]_{\varphi} & \cdots & [z_{1}, \overline{z_{n}}]_{\varphi} \\
		[z_{2}, \overline{z_{1}}]_{\varphi} & [z_{2}, \overline{z_{2}}]_{\varphi} & \cdots & [z_{2}, \overline{z_{n}}]_{\varphi} \\
		\vdots & \vdots & \ddots & \vdots \\
		[z_{n}, \overline{z_{1}}]_{\varphi} & [z_{n}, \overline{z_{2}}]_{\varphi} & \cdots &  [z_{n}, \overline{z_{n}}]_{\varphi}
	\end{bmatrix}
	\end{align*}
	is positive, then $\varphi$ is a Pick function.
\end{lause}

Note that if all the Pick matrices are positive, function clearly has non-negative imaginary part. Thus we ``only" need to verify analyticity.

Let's first check continuity. For this we only need positivity on $2 \times 2$-matrices.

\begin{lause}\label{pick_continuity_lemma}
	Let $A \subset \Hp$ and $\varphi : A \to \Hpc$ such that for any $z_{1}, z_{2} \in \Hp$ Pick matrix
	\begin{align*}
	\begin{bmatrix}
		[z_{1}, \overline{z_{1}}]_{\varphi} & [z_{1}, \overline{z_{2}}]_{\varphi}\\
		[z_{2}, \overline{z_{1}}]_{\varphi} & [z_{2}, \overline{z_{2}}]_{\varphi}\\
	\end{bmatrix}
	\end{align*}
	is positive. Then $f$ is locally Lipschitz, in particular continuous.
\end{lause}
\begin{proof}
	TODO
\end{proof}

Curiosly enough, $\varphi$ is analytic as long as all its $3 \times 3$ Pick matrices are positive. This result is known as Hindmarsh's theorem.

\begin{lause}\label{Hindmarsh_theorem}
	Let $U \subset \Hp$ be open and $\varphi : U \to \Hpc$ such that for every $z_{1}, z_{2}, z_{3} \in \Hp$ Pick matrix
	\begin{align*}
	\begin{bmatrix}
		[z_{1}, \overline{z_{1}}]_{\varphi} & [z_{1}, \overline{z_{2}}]_{\varphi} & [z_{1}, \overline{z_{3}}]_{\varphi}\\
		[z_{2}, \overline{z_{1}}]_{\varphi} & [z_{2}, \overline{z_{2}}]_{\varphi} & [z_{2}, \overline{z_{3}}]_{\varphi}\\
		[z_{3}, \overline{z_{1}}]_{\varphi} & [z_{3}, \overline{z_{2}}]_{\varphi} & [z_{3}, \overline{z_{3}}]_{\varphi}\\
	\end{bmatrix}
	\end{align*}
	is positive. Then $f$ is analytic.
\end{lause}
\begin{proof}
	TODO
\end{proof}

As an immediate corollary we get theorem \ref{pick-nevanlinna_converse}.

\begin{proof}[Proof of theorem \ref{pick-nevanlinna_converse}]
If all Pick matrices are positive, so are all $3 \times 3$ Pick matrices.
\end{proof}

We can hence characterize Pick functions purely with Pick matrices, without limits and concerns of regularity, ``weakly". As an immediate corollary we get proposition \ref{pick_convergence}.

\begin{proof}[Proof of theorem \ref{pick_convergence}]
Pointwise limits preserve positivity of the Pick matrices.
\end{proof}

One might still go even further and argue that one does not need Pick matrices larger than $3 \times 3$ to talk about Pick functions. They however carry interesting ``global" information.

\begin{lause}\label{open_pick_nevanlinna}
	Let $U \subset \Hp$ be open and $\varphi : U \to \Hpc$ such that for any $n \geq 1$ and $z_{1}, z_{2}, \ldots, z_{n} \in U$ the respective Pick matrix
	\begin{align*}
	\begin{bmatrix}
		[z_{1}, \overline{z_{1}}]_{\varphi} & [z_{1}, \overline{z_{2}}]_{\varphi} & \cdots & [z_{1}, \overline{z_{n}}]_{\varphi} \\
		[z_{2}, \overline{z_{1}}]_{\varphi} & [z_{2}, \overline{z_{2}}]_{\varphi} & \cdots & [z_{2}, \overline{z_{n}}]_{\varphi} \\
		\vdots & \vdots & \ddots & \vdots \\
		[z_{n}, \overline{z_{1}}]_{\varphi} & [z_{n}, \overline{z_{2}}]_{\varphi} & \cdots &  [z_{n}, \overline{z_{n}}]_{\varphi}
	\end{bmatrix}
	\end{align*}
	is positive. Then $\varphi$ is a restriction of an unique Pick function.
\end{lause}

This tells us that we may recognise Pick functions from local information. To prove the theorem, we first introduce the notion of \textit{Pick point}.

\begin{maar}
	Let $U \subset \Hp$ be open and $\varphi : U \to \Hpc$. We say that $z \in U$ is Pick point of $\varphi$, if $\varphi$ is analytic at $z$ and the $n \times n$ matrix
	\begin{align*}
	\begin{bmatrix}
		[z, \overline{z}]_{\varphi} & [z, \overline{z}, \overline{z} ]_{\varphi} & \cdots & [z, \overline{z}, \overline{z}, \ldots, \overline{z}]_{\varphi} \\
		[z, z, \overline{z}]_{\varphi} & [z, z, \overline{z}, \overline{z} ]_{\varphi} & \cdots & [z, z, \overline{z}, \overline{z}, \ldots, \overline{z}]_{\varphi} \\
		\vdots & \vdots & \ddots & \vdots \\
		[z, z, \ldots, z, \overline{z}]_{\varphi} & [z, z, \ldots, z, \overline{z}, \overline{z} ]_{\varphi} & \cdots & [z, z, \ldots, z, \overline{z}, \overline{z}, \ldots, \overline{z}]_{\varphi} \\
	\end{bmatrix}
	\end{align*}
	is positive for every $n$.
\end{maar}

Where does this definition come from? The idea is to answer the question: what does it mean the Pick matrices to be non-negative at a single point, say $z_{0}$? In the definition of Pick matrix we could let all the variables be equal and conclude that the matrix
\begin{align*}
\begin{bmatrix}
	[z_{0}, \overline{z_{0}}]_{\varphi} & [z_{0}, \overline{z_{0}}]_{\varphi} & \cdots & [z_{0}, \overline{z_{0}}]_{\varphi} \\
	[z_{0}, \overline{z_{0}}]_{\varphi} & [z_{0}, \overline{z_{0}}]_{\varphi} & \cdots & [z_{0}, \overline{z_{0}}]_{\varphi} \\
	\vdots & \vdots & \ddots & \vdots \\
	[z_{0}, \overline{z_{0}}]_{\varphi} & [z_{0}, \overline{z_{0}}]_{\varphi} & \cdots &  [z_{0}, \overline{z_{0}}]_{\varphi}
\end{bmatrix}
\end{align*}
is positive, but this would only tell us that $[z_{0}, \overline{z_{0}}]_{\varphi}$ is non-negative. We need derivatives.

The idea is to $*$-conjugate the Pick matrix first. Say $n = 2$. If we subtract first row from the second in the Pick matrix
\begin{align*}
\begin{bmatrix}
	[z_{1}, \overline{z_{1}}]_{\varphi} & [z_{1}, \overline{z_{2}}]_{\varphi}\\
	[z_{2}, \overline{z_{1}}]_{\varphi} & [z_{2}, \overline{z_{2}}]_{\varphi}\\
\end{bmatrix}
\end{align*}
we get
\begin{align*}
\begin{bmatrix}
	[z_{1}, \overline{z_{1}}]_{\varphi} & [z_{1}, \overline{z_{2}}]_{\varphi}\\
	[z_{2}, \overline{z_{1}}]_{\varphi} - [z_{1}, \overline{z_{1}}]_{\varphi} & [z_{2}, \overline{z_{2}}]_{\varphi} - [z_{1}, \overline{z_{2}}]_{\varphi}\\
\end{bmatrix}
=
\begin{bmatrix}
	[z_{1}, \overline{z_{1}}]_{\varphi} & [z_{1}, \overline{z_{2}}]_{\varphi}\\
	(z_{2} - z_{1})[z_{1}, z_{2}, \overline{z_{1}}]_{\varphi} & (z_{2} - z_{1})[z_{1}, z_{2}, \overline{z_{2}}]_{\varphi}
\end{bmatrix}.
\end{align*}
Now subtracting first column from the second results in
\begin{align*}
&\begin{bmatrix}
	[z_{1}, \overline{z_{1}}]_{\varphi} & [z_{1}, \overline{z_{2}}]_{\varphi} - [z_{1}, \overline{z_{1}}]_{\varphi}\\
	(z_{2} - z_{1})[z_{1}, z_{2}, \overline{z_{1}}]_{\varphi} & (z_{2} - z_{1})[z_{1}, z_{2}, \overline{z_{2}}]_{\varphi} - (z_{2} - z_{1})[z_{1}, z_{2}, \overline{z_{1}}]_{\varphi}
\end{bmatrix} \\
=&
\begin{bmatrix}
	[z_{1}, \overline{z_{1}}]_{\varphi} & \overline{(z_{2} - z_{1})}[z_{1}, \overline{z_{1}}, \overline{z_{2}}]_{\varphi}\\
	(z_{2} - z_{1})[z_{1}, z_{2}, \overline{z_{1}}]_{\varphi} & (z_{2} - z_{1}) \overline{(z_{2} - z_{1})}[z_{1}, z_{2}, \overline{z_{1}}, \overline{z_{2}}]_{\varphi}
\end{bmatrix}.
\end{align*}
In the language of matrices this really says that
\begin{align*}
&\begin{bmatrix}
	1 & 0 \\
	-1 & 1
\end{bmatrix}
\begin{bmatrix}
	[z_{1}, \overline{z_{1}}]_{\varphi} & [z_{1}, \overline{z_{2}}]_{\varphi}\\
	[z_{2}, \overline{z_{1}}]_{\varphi} & [z_{2}, \overline{z_{2}}]_{\varphi}\\
\end{bmatrix}
\begin{bmatrix}
	1 & -1 \\
	0 & 1
\end{bmatrix} \\
= &
\begin{bmatrix}
	[z_{1}, \overline{z_{1}}]_{\varphi} & \overline{(z_{2} - z_{1})}[z_{1}, \overline{z_{1}}, \overline{z_{2}}]_{\varphi}\\
	(z_{2} - z_{1})[z_{1}, z_{2}, \overline{z_{1}}]_{\varphi} & (z_{2} - z_{1}) \overline{(z_{2} - z_{1})}[z_{1}, z_{2}, \overline{z_{1}}, \overline{z_{2}}]_{\varphi}
\end{bmatrix} \\
= &
\begin{bmatrix}
	1 & 0 \\
	0 & (z_{2} - z_{1})
\end{bmatrix}
\begin{bmatrix}
	[z_{1}, \overline{z_{1}}]_{\varphi} & [z_{1}, \overline{z_{1}}, \overline{z_{2}}]_{\varphi}\\
	[z_{1}, z_{2}, \overline{z_{1}}]_{\varphi} & [z_{1}, z_{2}, \overline{z_{1}}, \overline{z_{2}}]_{\varphi}
\end{bmatrix}
\begin{bmatrix}
	1 & 0 \\
	0 & \overline{(z_{2} - z_{1})}
\end{bmatrix}:
\end{align*}
matrices
\begin{align*}
\begin{bmatrix}
	[z_{1}, \overline{z_{1}}]_{\varphi} & [z_{1}, \overline{z_{2}}]_{\varphi}\\
	[z_{2}, \overline{z_{1}}]_{\varphi} & [z_{2}, \overline{z_{2}}]_{\varphi}\\
\end{bmatrix}
\text{ and }
\begin{bmatrix}
	[z_{1}, \overline{z_{1}}]_{\varphi} & [z_{1}, \overline{z_{1}}, \overline{z_{2}}]_{\varphi}\\
	[z_{1}, z_{2}, \overline{z_{1}}]_{\varphi} & [z_{1}, z_{2}, \overline{z_{1}}, \overline{z_{2}}]_{\varphi}
\end{bmatrix}
\end{align*}
are congruent.

Generalizing this argument we see that the matrices
\begin{gather*}
\begin{bmatrix}
	[z_{1}, \overline{z_{1}}]_{\varphi} & [z_{1}, \overline{z_{2}}]_{\varphi} & \cdots & [z_{1}, \overline{z_{n}}]_{\varphi} \\
	[z_{2}, \overline{z_{1}}]_{\varphi} & [z_{2}, \overline{z_{2}}]_{\varphi} & \cdots & [z_{2}, \overline{z_{n}}]_{\varphi} \\
	\vdots & \vdots & \ddots & \vdots \\
	[z_{n}, \overline{z_{1}}]_{\varphi} & [z_{n}, \overline{z_{2}}]_{\varphi} & \cdots &  [z_{n}, \overline{z_{n}}]_{\varphi}
\end{bmatrix} \\
\text{ and } \\
\begin{bmatrix}
	[z_{1}, \overline{z_{1}}]_{\varphi} & [z_{1}, \overline{z_{1}}, \overline{z_{2}}]_{\varphi} & \cdots & [z_{1}, \overline{z_{1}}, \overline{z_{2}}, \ldots, \overline{z_{n}}]_{\varphi} \\
	[z_{1}, z_{2}, \overline{z_{1}}]_{\varphi} & [z_{1}, z_{2}, \overline{z_{1}}, \overline{z_{2}}]_{\varphi} & \cdots & [z_{1}, z_{2}, \overline{z_{1}}, \overline{z_{2}}, \ldots, \overline{z_{n}}]_{\varphi} \\
	\vdots & \vdots & \ddots & \vdots \\
	[z_{1}, z_{2}, \ldots, z_{n}, \overline{z_{1}}]_{\varphi} & [z_{1}, z_{2}, \ldots, z_{n}, \overline{z_{1}}, \overline{z_{2}}]_{\varphi} & \cdots &  [z_{1}, z_{2}, \ldots, z_{n}, \overline{z_{1}}, \overline{z_{2}}, \ldots, \overline{z_{n}}]_{\varphi}
\end{bmatrix}
\end{gather*}
are congruent. We hence conclude the following.

\begin{lem}
	Let $U \subset \Hp$ be open and $\varphi : U \to \Hpc$. Let $z_{0} \in U$. Assume that for some $r > 0$, for every $n \geq 1$ and $z_{1}, z_{2}, \ldots, z_{n} \in \D(z_{0}, r) \cap U$ the respective Pick matrix is positive. Then $z_{0}$ is Pick point of $\varphi$.
\end{lem}
\begin{proof}
	By theorem \ref{Hindmarsh_theorem} $\varphi$ is analytic at $z_{0}$. By previous observation all the matrices of the form
\begin{align}\label{extended_dobsch_matrix}
\begin{bmatrix}
	[z_{1}, \overline{z_{1}}]_{\varphi} & [z_{1}, \overline{z_{1}}, \overline{z_{2}}]_{\varphi} & \cdots & [z_{1}, \overline{z_{1}}, \overline{z_{2}}, \ldots, \overline{z_{n}}]_{\varphi} \\
	[z_{1}, z_{2}, \overline{z_{1}}]_{\varphi} & [z_{1}, z_{2}, \overline{z_{1}}, \overline{z_{2}}]_{\varphi} & \cdots & [z_{1}, z_{2}, \overline{z_{1}}, \overline{z_{2}}, \ldots, \overline{z_{n}}]_{\varphi} \\
	\vdots & \vdots & \ddots & \vdots \\
	[z_{1}, z_{2}, \ldots, z_{n}, \overline{z_{1}}]_{\varphi} & [z_{1}, z_{2}, \ldots, z_{n}, \overline{z_{1}}, \overline{z_{2}}]_{\varphi} & \cdots &  [z_{1}, z_{2}, \ldots, z_{n}, \overline{z_{1}}, \overline{z_{2}}, \ldots, \overline{z_{n}}]_{\varphi}
\end{bmatrix}
\end{align}
are positive. By letting $z_{1}, z_{2}, \ldots, z_{n} \to z_{0}$ we get the claim.
\end{proof}
Matrix \ref{extended_dobsch_matrix} is called \textit{extended Pick matrix}.

The idea of the proof of theorem \ref{open_pick_nevanlinna} is the following: we try to make every point in $\Hp$ is Pick point of $\varphi$. For this we need two lemmas. first one says that if point is a Pick point of $\varphi$, then $\varphi$ can be extended to a reasonable large disc around $z_{0}$, which is really say that the Taylor coefficients of $\varphi$ at $z_{0}$ don't grow too fast. Second one tells us that if managed to extend function to such disc, then all the points in the disc are Pick points.

\begin{lem}
	There exists absolute constant $c_{0}$ with the following property: Let $U \subset \Hp$ be open and $\varphi : U \to \Hpc$. Assume that $z_{0} \in U$ is a Pick point of $U$. Then the Taylor series of $\varphi$ at $z_{0}$ converges in $\D(z_{0}, c_{0} \Im(z_{0}))$. One may take $c_{0} = ?$.
\end{lem}
\begin{proof}
	TODO
\end{proof}

\begin{lem}\label{pick_point_lemma}
	Let $U \subset \Hp$ be open and $\varphi : U \to \Hpc$. Let $z_{0} \in U$ and $r > 0$. Assume that $z_{0}$ is Pick point of $\varphi$ and $\varphi$ is analytic in $\D(z_{0}, r)$. Then all Pick matrices of $\varphi$ are positive on $\D(z_{0}, r)$. Consequently, all the points in $\D(z_{0}, r)$ are Pick points of $z_{0}$.
\end{lem}
\begin{proof}
	TODO.
\end{proof}

\begin{proof}[Proof of theorem \ref{open_pick_nevanlinna}]
	Consider all open sets $U \subset V \subset \Hp$ such that $\varphi$ may be extended to $V$ so that all the points of $V$ are Pick points of $\varphi$ (or the extension thereof). These sets trivially satisfy conditions of Zorn's lemma (where partial order is given by inclusion) so we may Pick maximal such set, $V$. We claim that this set is $\Hp$. If not, we may pick a point $z_{0}$ in $\Hp \cap \partial V$. Now pick $z \in V$ such that $|z - z_{0}| < \Im(z_{0}) c_{0}$: we may extend $\varphi$ now further to $\D(z_{0}, c_{0} \Im(z_{0}))$ by Taylor series and all points in the extension are Pick points, which contradicts the maximality.
\end{proof}

There is fundamental flaw with the second argument: when we extend $\varphi$ with Taylor series, how do we know that these extensions are consistent? If original set $U$ was, say, disjoint union of two discs, and Taylor series centered at a point of one disc would converge also at (some part of) other disc, how do we know that the Taylor series converges to the predefined values in the one disc? This is not at all clear.

If $U$ is disc or more generally domain, we can use the monodromy theorem to show that this kind of extension is indeed possible. Indeed, upper half-plane is simply connected and by the previous lemmas we can continue $\varphi$ along any path. This, however, still doesn't fix the problem with two disjoint discs.

We can salvage the proof by improving the second lemma. We have to somehow remember the information of the positivity of all the Pick matrices, otherwise we might run into problem. Let's have notion for this.

\begin{maar}
	Let $U \subset \Hp$ be open and $\varphi : U \to \Hpc$. We call $\varphi$ \textit{weakly Pick}, or say it is \textit{weak Pick function} if all its Pick matrices are positive.
\end{maar}

\begin{lem}
	Let $U \subset \Hp$ be open and $\varphi : U \to \Hpc$ weakly Pick. Assume that for some $z_{0} \in U$ and $r > 0$ the Taylor series of $\varphi$ at $z_{0}$ converges in $\D(z_{0}, r)$. Then the values of Taylor series in $\D(z_{0}, r) \cap U$ coincide with $\varphi$ and the resulting extension is weakly Pick in $U \cup \D(z_{0}, r)$.
\end{lem}
\begin{proof}
	TODO
\end{proof}

\begin{proof}[Fix of the proof of theorem \ref{open_pick_nevanlinna}]
Now the Zorn's lemma works if we change the condition a bit: we look at all the extensions of $\varphi$ to open subsets of upper half-plane for which all the Pick matrices are positive.

Zorn's lemma is not really necessary here: one could write explicit extension scheme (TODO: picture) and the lemmas would guarantee that we can always both extend further and extensions are always consistent.
\end{proof}

\section{Pick-Nevanlinna Interpolation theorem}

One can still considerably strenghten theorem \ref{open_pick_nevanlinna}: instead of open set, domain of $\varphi$ could be any set. Then we don't in general have unique extension. This is the content of Pick-Nevanlinna interpolation theorem.

\begin{lause}[Pick-Nevanlinna interpolation theorem]\label{pick_nevanlinna_interpolation}
	Let $A \subset \H_{+}$ and $\varphi \to \Hpc$ weakly Pick. Then $\varphi$ is a restriction of a Pick function to $A$.
\end{lause}

Note that the notion of weak Pick functions makes perfect sense in this more general setting. In the general setting we can't use Taylor series to extend the function anymore, but it turns out that this is not too big of a problem. The idea is again to extend $\varphi$ to larger and larger sets. Note that to do this we only need to be able to extend $\varphi$ to one new point so that all Pick matrices are again positive: we can then use the Zorn's lemma -trick again.

\begin{lem}\label{pick_extension_lemma}
	Let $A \subset \H_{+}$ and $\varphi \to \Hpc$ weakly Pick. Then if $z_{0} \in \Hp \setminus A$ there exists $w_{0} \in \Hpc$ such that if we extend $\varphi$ to $z_{0}$ by setting $\varphi(z_{0}) = w_{0}$, also the extension is weakly Pick.
\end{lem}
\begin{proof}
	Let's first consider the case of finite $A$. TODO

	Let's now consider case of arbitrary $A$. For every finite subset $F \subset A$ we know that the set of suitable $w_{0}$'s, say $W_{F}$, is a closed ball. We also know that $W_{F_{1} \cup F_{2}} \subset W_{F_{1}} \cap W_{F_{2}}$. But this means that the family
	\begin{align*}
		\{W_{F} \subset \Hp | \text{$F$ is finite (non-empty) subset of $A$} \}
	\end{align*}
	has finite intersection property, and since it's members are compact, we know that they have non-empty intersection. This intersection is the place where we find the $w_{0}$.
\end{proof}

\begin{proof}[Proof of theorem \ref{pick_nevanlinna_interpolation}]
Let us consider the set of all weakly Pick extensions of $\varphi$, ordered by restriction. This family trivially satisfies conditions of the Zorn's lemma, so there's a maximal element, $\varphi_{0}$. But by the previous lemma $\varphi_{0}$ must be defined in the whole upper half-plane, since if not, we could extend it to one more point. Finally, by theorem \ref{pick-nevanlinna_converse} the resulting map is Pick function.
\end{proof}

Again, one doesn't really need the Zorn's lemma.

\begin{proof}[Alternate proof of theorem \ref{pick_nevanlinna_interpolation}]
	If $\Hp \setminus A$ is finite, simply apply lemma \ref{pick_extension_lemma} repeatedly. If not, we may pick a countable dense set in $C \subset \Hp \setminus A$ and extend $\varphi$ there. As a result we get a map in a dense subset of upper half-plane, which is continuous by lemma \ref{pick_continuity_lemma}. This means that we have continuous extension to whole upper half-plane, and by continuity also all the Pick matrices of the extension are positive. Finally, by theorem \ref{pick-nevanlinna_converse} the final extension is a Pick function.
\end{proof}

\section{Schur transform}

Properties of Pick functions translate nicely to those of positive maps. Most obvious of these properties is the cone structure: Pick matrix is linear in the function. The automorphims of the upper half-plane correspond to one dimensional projections. Indeed, Pick matrix of map of the form $z \mapsto \frac{a z + b}{c z + d}$ is given by
\begin{align*}
	\left[\frac{\frac{a z_{i} + b}{c z_{i} + d} - \frac{a \overline{z_{j}} + b}{c \overline{z_{j}} + d}}{z_{i} - \overline{z_{j}}}\right]_{i, j = 1}^{n}
	=
	\left[\frac{a d - bc}{(c z_{i} + d) (c \overline{z_{i}} + d)}\right]_{i, j = 1}^{n},
\end{align*}
which is of rank $1$. Composition corresponds to Hadamard product: if $\varphi = \varphi_{2} \circ \varphi_{1}$, we have
\begin{align*}
	\left[\frac{\varphi(z_{i}) - \overline{\varphi(z_{j})}}{z_{i} - \overline{z_{j}}}\right]_{i, j = 1}^{n}
	=
	\left[\frac{\varphi_{2}(\varphi_{1}(z_{i})) - \overline{\varphi_{2}(\varphi_{1}(z_{j}))}}{\varphi_{1}(z_{i}) - \overline{\varphi_{1}(z_{j})}}\right]_{i, j = 1}^{n}
	\circ
	\left[\frac{\varphi_{1}(z_{i}) - \overline{\varphi_{1}(z_{j})}}{z_{i} - \overline{z_{j}}}\right]_{i, j = 1}^{n}.
\end{align*}

There's however more subtle connection, one between Schur transform and, rather appropriately, Schur complement.

If $\psi$ is Schur function such that $\psi(0) = 0$, then, by the Schwarz lemma, also $\psi(z)/z$ is a Schur function. One may again translate this to Pick functions, and get an interesting corollary: if $\varphi$ is a Pick function such that $\varphi(i) = i$, then also
\begin{align*}
	\frac{\varphi(z) - z}{1 + z \varphi(z)}
\end{align*}
is a Pick function. Actually, this gives a bijection between Pick functions and Pick functions with $\varphi(i) = i \ldots$ almost: $z \mapsto -\frac{1}{z}$ would like to map to constant infinity.

We could form similar bijection for any pair $(z, w) \in \H_{+}^{2}$: Pick functions for which $\varphi(z) = w$.

The previous bijection translates nicely to Pick matrices. Take sequence of points $z_{1}, z_{2}, \ldots, z_{n} \in \H_{+} \setminus \{i\}$. Theorem \ref{pick-nevanlinna_theorem} for $\varphi$ and points $i, z_{1}, \ldots, z_{n}$ implies that the matrix
\begin{align*}
	\begin{bmatrix}
		[i, -i]_{\varphi} & [i, \overline{z_{1}}]_{\varphi} & \cdots & [i, \overline{z_{n}}]_{\varphi} \\
		[z_{1}, -i]_{\varphi} & [z_{1}, \overline{z_{1}}]_{\varphi} & \cdots & [z_{1}, \overline{z_{n}}]_{\varphi} \\
		\vdots & \vdots & \ddots & \vdots \\
		[z_{n}, -i]_{\varphi} & [z_{n}, \overline{z_{1}}]_{\varphi} & \cdots &  [z_{n}, \overline{z_{n}}]_{\varphi}
	\end{bmatrix}
\end{align*}
is positive. Since $[i, -i]_{\varphi} = 1$ is positive, taking Schur-complement with respect to upper-left corner this is equivalent to the matrix
\begin{align*}
	& \left([z_{i}, \overline{z_{j}}]_{\varphi} - [z_{i}, -i]_{\varphi}[i, \overline{z_{j}}]_{\varphi}\right)_{1 \leq i, j \leq n} \\
	=& \left(\left(\frac{\varphi(z_{i})- \overline{\varphi(z_{j})}}{z_{i} - \overline{z_{j}}}\right) - \left(\frac{\varphi(z_{i}) + i}{z_{i} + i}\right)\left(\frac{i - \overline{\varphi(z_{j})}}{i - \overline{z_{j}}}\right)\right)_{1 \leq i, j \leq n} \\
	=& \left(\frac{[z_{i}, \overline{z_{j}}]_{\varphi} (z_{i} \overline{z_{j}} + 1) - 1 - \varphi(z_{i}) \overline{\varphi(z_{j})}}{(z_{i} + i) (\overline{z_{j}} - i)}\right)_{1 \leq i, j \leq n}
\end{align*}
being positive. But if one applies the theorem to the function $\frac{\varphi(z) - z}{1 + z \varphi(z)}$ and points $z_{1}, z_{2}, \ldots, z_{n}$, one arrives at the matrix
\begin{align*}
	\left(\frac{[z_{i}, \overline{z_{j}}]_{\varphi} (z_{i} \overline{z_{j}} + 1) - 1 - \varphi(z_{i}) \overline{\varphi(z_{j})}}{(1 + z_{i} \varphi(z_{i}))(1 + \overline{z_{j}} \overline{\varphi(z_{j})})}\right)_{1 \leq i, j \leq n}.
\end{align*}
Two resulting matrices are evidently congruent.

This line of thinking leads to alternate proof for the Pick-Nevanlinna interpolation theorem for finite domain $A$.

\begin{proof}[Alternate proof of \ref{pick_nevanlinna_interpolation} for finite $A$]
	We proceed by induction. If $z_{1} = i = w_{1}$, we just noted that this matrix being positive is equivalent to smaller matrix, namely
	\[
		\left(\frac{\frac{w_{i} - z_{i}}{1 + z_{i} w_{i}} - \frac{\overline{w_{i}} - \overline{z_{i}}}{1 + \overline{z_{i}} \overline{w_{i}}}}{z_{i} - \overline{z_{j}}}\right)_{2 \leq i, j \leq n}
	\]
	being positive. By inductive hypothesis this then means that there exists a Pick function $\tilde{\varphi}$ with $\tilde{\varphi}(z_{i}) = \frac{w_{i} - z_{i}}{1 + z_{i} w_{i}}$. But then $\varphi(z) = \frac{\tilde{\varphi}(z) + z}{1 - z \tilde{\varphi}(z)}$ fits the bill. TODO: general case (introduce Schur transform)
\end{proof}

\section{Compactness}

One can lift the previous argument of the Pick-Nevanlinna interpolation theorem to infinite sets by the following compactness result.

\begin{lause}\label{pick_compactness}
	Let $F = \{\varphi_{j} | j \in J \}$ be a family of Pick functions uniformly bounded in a point (or upper half-plane). Then $F$ is compact under the topology of pointwise convergence.
\end{lause}
\begin{proof}
	Note that respective claim holds for Schur functions by Montel's theorem, so hence also holds for Pick functions $\ldots$ with one small cavaet. The problem is that Schur function constant $-1$ corresponds to constant $\infty$, which isn't proper Pick function. But by the boundedness condition this cannot happen.
\end{proof}

There's really nothing special about Pick functions here: in the same way one could prove any family of analytic maps with common (non-whole-of-$\C$) simply connected domain is compact. Of course we even get locally uniform convergence but that won't be important for us.

One could also give weaker proof.

\begin{proof}[Alternate proof for the theorem \ref{pick_compactness}]
	Fix sequence in $F$. By lemma \ref{pick_continuity_lemma} and the condition values $\sup_{z \in K}|\varphi_{j}(z)|$ are uniformly bounded for every compact set. By Arzel\`{a}-Ascoli theorem for every compact set $K$ we can find a subsequence convergent in $K$ and by taking exhaustin upper half-plane by nested compact sequence and taking the diagonal sequence we get the claim. By proposition \ref{pick_convergence} the limit is also Pick function.
\end{proof}

Of course, one could use the previous approach to prove the Montel's theorem in the first place, but the idea is that we can also forget the analyticity and work with Pick matrices on the weak level.

\section{TODO}
\begin{itemize}
	\item Poincaré metric: discs are discs, Apollonius circle
	\item Pick-Nevanlinna-Herglotz representation theorem
	\item Examples of representing measures behind functions and functions behind representing measures
	\item Spectral commutant lifting theorem
	\item Use Morera's theorem to prove weak Hindmarsh's theorem
\end{itemize}

