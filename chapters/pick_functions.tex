\chapter{Pick-Nevanlinna functions}

\textit{Pick-Nevanlinna function} is an analytic function defined in upper half-plane with a non-negative imaginary part. Such functions are sometimes also called Herglotz or $\R$ functions; we will call them just \textit{Pick functions}. The class of Pick functions is denoted by $\pickclass$.

\section{Examples and basic properties}

Most obvious examples of Pick functions might be functions of the form $\alpha z + \beta$ where $\alpha, \beta \in \R$ and $\alpha \geq 0$. Of course one could also take $\beta \in \Hpc$. As non-constant analytic functions are open mappings, real constants are the only Pick functions failing to map $\Hp \to \Hp$.

Pick functions can be thought of a set of ``positive analytic functions".

\begin{lause}
	$\pickclass \subset \{\text{analytic maps on $\Hp$} \}$ is a closed convex cone.
\end{lause}
\begin{proof}
	Again, t.f.i.f \ref{positive_machine}.
\end{proof}

Also a composition of Pick functions is a Pick function.

The map $z \mapsto -\frac{1}{z}$ is evidently a Pick function. Hence are also all functions of the form
\begin{align*}
	\alpha z + \beta + \sum_{i = 1}^{N} \frac{m_{i}}{\lambda_{i}- z},
\end{align*}
where $N$ is non-negative integer, $\alpha, m_{1}, m_{2}, \ldots, m_{N} \geq 0$, $\beta \in \Hp$ and $\lambda_{1}, \ldots, \lambda_{N} \in \Hm$.

There are (luckily) more interesting examples. All the functions of the form $x^{p}$ (with natural branch) where $0 < p < 1$ are Pick functions; similarly for $\log$. Another classic example is $\tan$. Indeed, by the addition formula
\begin{eqnarray*}
	\tan(x + i y) &=& \frac{\tan(x) + \tan(i y)}{1 - \tan(x) \tan(i y)} = \frac{\tan(x) + i \tanh(y)}{1 - i \tan(x) \tanh(y)} \\
	&=& \frac{\tan(x)(1 + \tanh^2(y))}{1 + \tan^2(x) \tanh^2(y)} + i \frac{(1 + \tan^2(x))\tanh(y)}{1 + \tan^2(x) \tanh^2(y)},
\end{eqnarray*}
and $y$ and $\tanh(y)$ have the same sign.

$\pickclass$ is almost salient: if $\varphi$ is analytic and $\Im(\varphi) = 0$, then $\varphi$ is a real constant (by Cauchy-Riemann equations, for instance). And again, this suggests that one should think about Pick functions up to a real constant.

So far we have made no mention on the topology, as it's usually taken to be the topology of locally uniform convergence. This definitely works (as it makes the evaluation functionals continuous), but we can do much better. It namely turns out that we can consider the set of Pick functions as a proper cone of $\C^{\Hp}$, set of all functions, with the topology of pointwise convergence.

\begin{prop}\label{pick_convergence}
	If $(\varphi_{i})_{i = 1}^{\infty}$ is a sequence of Pick functions converging pointwise, the limit function is also a Pick function.
\end{prop}

This result is far from clear: pointwise limits of analytic functions need not analytic in general. We will not prove the result yet, but it strongly suggests that there is something more going on; Pick functions are very rigid. Note also that if Pick functions are thought of a subset of all functions, the definition of the cone doesn't really fit the general framework of theorem \ref{positive_machine}. This suggests that question one should ask is:

\begin{quest}\label{pick_predual}
	What is the ``correct" predual for $\pickclass$?
\end{quest}


\section{Rigidity}

\subsection{Boundary}

To understand the rigidity phenomena we take a brief look at a close relative to Pick functions, \textit{Schur functions}. Schur funtions are analytic maps from open unit disc to closed unit disc. Classic fact about these functions is the Schwarz lemma.

\begin{lause}[Schwarz lemma]
	Let $\psi : \D \to \D$ be analytic such that $\psi(0) = 0$. Then $|\psi(z)| \leq |z|$ for any $z \in \D$.
\end{lause}

The textbook proof is based on two observations about analytic functions.
\begin{itemize}
	\item If $\varphi$ is analytic at $a$ with $\varphi(a) = 0$, then $\varphi/(\cdot - a)$ is also analytic.
	\item If $\varphi$ is analytic on closed unit disc and $|\varphi| \leq 1$ on the boundary of the disc, then $|\varphi| \leq 1$ inside the disc.
\end{itemize}

The first observation might not be very surprising, and it holds for smooth functions also. The second, on the other hand, is a true manifestation of the nature of the analytic maps: we can bound analytic functions simply by bounding them on the boundary of the domain. More generally: one knows everything about an analytic function on a domain simply by knowing it on a boundary, by Cauchy's integral formula.

This suggests that we should be able to recognize also Pick functions looking only at their boundary values. Actually even more is true: it suffices to look at the imaginary parts.

\begin{prop}
	Let $\varphi : U \to \C$ be analytic, such that $\overline{\Hp} \subset U$, and $\varphi$ is continuous at $\infty$. Then if the imaginary part of $\varphi$ is non-negative on the real axis, $\varphi$ is Pick function.
\end{prop}
\begin{proof}
	t.f.i.f the minimum principle applied to the harmonic function $\Im(\varphi)$.
\end{proof}

\subsection{Integral representations}

Recall that imaginary part of an analytic function determines also its real part, up to a constant, so we can also recover the function itself. This can be also done explicitly.

\begin{lause}
	Let $\varphi : U \to \C$ be analytic, such that $\overline{\Hp} \subset U$, and $\varphi(z) = O(|z|^{-\varepsilon})$ for some $\varepsilon > 0$ at infinity. Then for any $z \in \Hp$ we have
	\begin{align*}
		\varphi(z) = \frac{1}{\pi}\int_{\R} \frac{\Im(\varphi)(\lambda)}{\lambda - z} d \lambda
	\end{align*}
\end{lause}
\begin{proof}
	Note that the integral defines an analytic function, imaginary part of which equals
	\begin{align*}
		\frac{\Im(z)}{\pi}\int_{\R} \frac{\Im(\varphi)(\lambda)}{(\lambda - z)(\lambda - \overline{z})} d \lambda.
	\end{align*}
	This expression however equals $\Im(\varphi(z))$ by Poisson integral formula. By letting $z \to \infty$ one sees that also the real constants match.

	Alternatively one could observe that for a closed counter clockwise oriented curves $\gamma$ on the upper half-plane, enclosing $z$, we have
	\begin{align*}
		\varphi(z) = \frac{1}{2\pi i}\int_{\gamma} \frac{\varphi(\lambda)}{\lambda - z} d \lambda.
	\end{align*}
	Now given the bound, we may deform the contour to real axis. By comparing this identity and our goal, we are left to prove that
	\begin{align*}
		\frac{1}{2\pi i}\int_{\gamma} \frac{\overline{\varphi(\lambda)}}{\lambda - z} d \lambda = \frac{1}{2\pi i} \overline{\int_{\gamma} \frac{\varphi(\lambda)}{\lambda - \overline{z}} d \lambda}.
	\end{align*}
	But this is clear as $\varphi/(\cdot - \overline{z})$ is analytic in the upper half-plane.
\end{proof}

There's of course nothing really special about the decay assumption $\varphi(z) = O(|z|^{-\varepsilon})$; it's there just to make everything converge.

One can guarantee the convergence also by other means. Note that as the integrand behaves like $(\lambda - z)^{-1}$, if we subtract from it something (not depending on $z$) behaving the same way at the infinity, we ought to improve convergence, but only change the value of the function by a constant. As an example, consider the integral
\begin{align}\label{easy_pick_repr}
	\frac{1}{\pi}\int_{\R} \left(\frac{1}{\lambda - z} - \frac{[|\lambda| > 1]}{\lambda} \right) \Im(\varphi)(\lambda)d \lambda.
\end{align}
It converges to an analytic function as long as, say, $\Im(\varphi)$ is bounded. As before, its imaginary part coincides with $\varphi$'s so the functions are equal up to a real constant. Now, however, there's no reason for the real constants to match and indeed they need not.

Note that the previous idea could be used to construct Pick functions. Everything still makes sense if we replace $\Im(\varphi)$ by some other positive function, as long as the integral converges. Heck, we could replace it by any positive measure for which $\mu((\lambda^2 + 1)^{-1}) < \infty$.

(Almost) all the examples given before are actually just special cases of this construction. The rational functions $\frac{1}{\lambda - z}$, where $\lambda \in \R$ are obtained by setting $\mu = \delta_{\lambda}$. The power functions are obtained as
\begin{align*}
	z^{p} &= 1 + \frac{1}{\pi}\int_{-\infty}^{0} \left(\frac{1}{\lambda - z} - \frac{1}{\lambda - 1}\right) \Im(\lambda^{p}) d \lambda \\
	&=1 + \frac{1}{\pi}\int_{-\infty}^{0} \left(\frac{1}{\lambda - z} - \frac{1}{\lambda - 1}\right) |\lambda|^{p} \sin(\pi p) d \lambda,
\end{align*}
Logarithm is even simpler:
\begin{align*}
	\log(z) = \int_{-\infty}^{0} \left(\frac{1}{\lambda - z} - \frac{1}{\lambda - 1}\right) d \lambda.
\end{align*}
Tangent function could be obtained by putting $\delta$-measures to its poles, the points of the form $\frac{\pi}{2} + n \pi$, where $n \in \Z$.

The only exception is the function $z \mapsto \alpha z$ -- it can't be expressed as such integral. But even this failure is really more about poor point of view, as we will see in a minute. With these observations in mind it ought to be not too surprising that we have the following.

\begin{lause}\label{pick_nevanlinna_herglotz_representation_theorem}
	$\varphi \in \pickclass$, if and only
	\begin{align}\label{pick_representation}
		\varphi(z) = \alpha z + \beta + \int_{-\infty}^{\infty} \left(\frac{1}{\lambda - z} - \frac{\lambda}{\lambda^2 + 1}\right) d \mu(\lambda)
	\end{align}
	for some $\alpha \geq 0$ and $\beta \in \R$ and a Radon measure $\mu$ with $\int_{-\infty}^{\infty} (\lambda^2 + 1)^{-1} d \mu(\lambda) < \infty$.
\end{lause}

Choosing $\lambda \mapsto \frac{\lambda}{\lambda^2 + 1}$ is common choice in the literature and is convenient as
\begin{itemize}
	\item It's real, so the integrand is Pick function for any $\lambda \in \R$.
	\item We may recover the constant $\beta$ as $\Re(\varphi(i))$.
\end{itemize}

To better explain the appearance of the linear term, we can write the integral in a sligtly different form. Denoting $d \nu(\lambda) = \frac{d \mu(\lambda)}{\lambda^2 + 1}$, the formula reads
\begin{align*}
	\varphi(z) = \alpha z + \beta + \int_{-\infty}^{\infty} \frac{\lambda z + 1}{\lambda - z} d \nu(\lambda).
\end{align*}
Here $\nu$ is just a finite Borel measure. Now it kind of makes sense to extend the domain of this measure to infinity: the linear term merely corresponds to $\delta$-measure at infinity point. Of course, should one formalize this line of thought, the question on the type of extended real line had to be asked and one should address the topology. The answer is that one should glue the real line into a circle. One shouldn't worry about such issues, though, as these thoughts are here merely for intuition. The giveaway is that $\alpha$ should be really thought as a part of the measure $\mu$, even though this might not make perfect sense.

We will not prove theorem \ref{pick_nevanlinna_herglotz_representation_theorem}, but it shall work as a motivation. Instead, we prove somewhat weaker claim.
\begin{lem}\label{pick_dense}
	Under the topology of pointwise convergence one has
	\begin{align*}
		\pickclass \subset \overline{\vspan}\{(\lambda - z)^{-1} | \lambda \in \R\}.
	\end{align*}
\end{lem}
\begin{proof}
	Denote the closure of the span by $\pickclass_{e}$. As $\lim_{\lambda \to \pm \infty}|\lambda|(\lambda - z)^{-1} = \pm 1$, $\R \in \pickclass_{e}$ Now by \ref{easy_pick_repr} all the bounded Pick functions extending analytically over $\R$ are also in $\pickclass_{e}$. We finish the proof by showing that such functions are dense in the set of all Pick functions: we denote this class by $\pickclass_{b}$.

	It is straightforward to check that
	\begin{align*}
		g_{\varepsilon}(z) = \frac{z + i \varepsilon}{1 - i \varepsilon z} \in \pickclass_{b}
	\end{align*}
	for any $\varepsilon > 0$. But now for any $\varphi \in \pickclass$ and $\varepsilon > 0$ also $\varphi \circ g_{\varepsilon} \in \pickclass_{b}$. Finally $\varphi = \lim_{\varepsilon \to 0} \varphi \circ g_{\varepsilon} \in \pickclass_{b}$, as we wanted.
\end{proof}

\section{Weakly Pick functions}
\subsection{Pick functionals}

Theorem \ref{pick_nevanlinna_herglotz_representation_theorem} suggests also an answer to the question \ref{pick_predual}: linear functionals on $\pickclass$ should be thought of some kind of rational functions.
\begin{maar}
	Let $X \subset \Hp$. We will denote by $R(X)$ the class of rational functions $r$ such that
	\begin{itemize}
		\item All poles of $r$ are simple and lie in $X \cup X^{*} = X \cup \{z \in \C | \overline{z} \in X\}$.
		\item $r(z) = O(|z|^{-2})$ at infinity.
	\end{itemize}
	We will also write $R_{+}(X)$ for the functions in $R(X)$, which are non-negative on $\R$.
\end{maar}
With the norm
\begin{align*}
	\|r\|_{R} = \sup_{x \in \R} |r(x)| (x^2 + 1)
\end{align*}
$R(X)$ becomes a topological vector space (over $\R$ or $\C$) and $R_{+}(X)$ its closed convex cone. Similarly to the previous chapter we have the dual pairing $\langle f, r \rangle_{L}$ between $\C^{X}$ and $R(X)$. Indeed, we may set
\begin{align*}
	\langle f, (z - a)^{-1} \rangle_{L} =
	\begin{cases}
		f(a) & \text{ if $a \in X$} \\
		\overline{f(\overline{a})} & \text{ if $a \in X^{*}$}
	\end{cases}
\end{align*}
and extend linearly.

\begin{lause}\label{pick_functionals}
	$R_{+}(\Hp)$ is the dual cone of $\pickclass$ in the sense of $\langle \cdot, \cdot \rangle$. In other words: all continuous linear functionals on $p^{*} \in \C^{\Hp}$ such that $p^{*}(\varphi) \geq 0$ for any $\varphi \in \pickclass$ are of the form $p^{*}(\varphi) = \langle \varphi, r \rangle_{L}$ for some $r \in R_{+}(\Hp)$.
\end{lause}
\begin{proof}
	$\pickclass^{*} \subset R_{+}(\Hp)$: It is well known that continuous dual of a product is direct sum dual (see for instance TODO). In other words elements of $\pickclass$ are finitely supported (i.e. finite linear combinations of evaluation functionals)\footnote{This fact is not important to us and we could have restricted our attention to finitely supported functionals anyway.}. We may hence interpret $p^{*} \in \pickclass^{*}$ as an rational function, $r$, with poles on $\Hp \cup \Hm$ and $r(\infty)$, acting with $\langle \cdot, r \rangle_{L}$. It is easy to see that $r \in R_{+}(\Hp)$ if we can verify that $r$ is non-negative. But as $(\lambda - z)^{-1} \in \pickclass$ for any $\lambda \in \R$, we have
	\begin{align*}
		p^{*}((\lambda - z)^{-1}) = \langle (\lambda - z)^{-1}, r \rangle_{L} = r(\lambda) \geq 0
	\end{align*}
	for any $\lambda \in \R$, as desired.

	$R_{+}(\Hp) \subset \pickclass^{*}$: As by the previous part $\langle (\lambda - z)^{-1}, r \rangle_{L}$ for any $\lambda \in \R$ and $r \in R_{+}(\Hp)$, we are done by Lemma \ref{pick_dense}.
\end{proof}

The rational functions $R_{+}(X)$ (identified with the respective linear maps $f \mapsto \langle f, r \rangle_{L}$) are called \textbf{Pick functionals} (on $X \subset \Hp$).

One can also interpret the Pick functionals as divided differences. Indeed, if $r \in R_{+}(X)$, it is easy to see that
\begin{align*}
	r(z) = \frac{N(q)}{|z - z_{1}|^2 \cdots |z - z_{n}|^2}
\end{align*}
for some $n \geq 1$, $q \in \C_{n - 1}[x]$ and pairwise distinct $z_{1}, z_{2}, \ldots, z_{n} \in X$. But this means that the respective Pick functional is given by
\begin{align*}
	\varphi \mapsto [z_{1}, \overline{z_{1}}, \ldots, z_{n}, \overline{z_{n}}]_{\varphi N(q)}.
\end{align*}

\subsection{Weakly Pick functions}

While Theorem \ref{pick_functionals} implies that $R_{+}(\Hp)$ is the dual cone of $\pickclass$, it turns out that it is also the predual we were looking for\footnote{While preduals are not unique in general, as $R_{+}(\Hp)$ is the dual cone of $\pickclass$, if $R_{+}(\Hp)$ is a predual of $\pickclass$, is unique maximal predual.}. This is far from clear: if $\varphi \in \C^{\Hp}$ is such that $\langle \varphi, r \rangle_{L} \geq 0$ for any $r \in R_{+}(\Hp)$, we only know that some kind of linear combinations of evaluations of $\varphi$ are non-negative. Where does the analyticity come from? Then again, in view of the theory of divided differences, particularly Theorems \ref{k-tone_smooth} and \ref{bernstein_theorem}, we have already seen this kind of phenomena.

\begin{maar}
	We will denote
	\begin{align*}
		\pickclass(X) = (R_{+}(X))^{*} = \{\varphi \in \C^{X} | \langle \varphi, r \rangle_{L} \geq 0 \text{ for any $r \in R_{+}(X)$}\}.
	\end{align*}
	and call elements of $\pickclass(X)$ \textbf{weakly Pick functions} (on $X$).
\end{maar}

With this terminology the claim ``$R_{+}(\Hp)$ is the predual of"

\begin{lause}
	We have $\pickclass(\Hp) = \pickclass$ i.e. all weakly Pick functions are Pick functions.
\end{lause}

Note that $\pickclass \subset \pickclass(\Hp)$ follows from \ref{pick_functionals}.

TODO

The other direction is tricky. Note that weakly Pick maps map to the upper half-plane so the interesting part is to prove that weakly Pick maps are analytic. For this we are going to verify bounds for the divided differences of $\varphi$. Recall that by theorem \ref{bounded_div} it suffices to verify that the order $2$ divided differences are locally bounded to prove that $\varphi$ is (continuously) differentiable. Strictly speaking we only proved the result on real line, but the proof would be almost identical in the complex case.

The idea is the following: we are going to formulate everything in terms of linear functionals. This idea is best illustrated with an example.

\begin{lem}[Harnack inequality]\label{pick_harnack_lemma}
	Let $\varphi$ be a weakly Pick function. Then for every compact $K \subset \Hp$ there exists a constant $C_{K}$ such that
	\begin{align*}
		\frac{\Im(\varphi(z))}{\Im(z)} \leq C_{K} \frac{\Im(\varphi(w))}{\Im(w)}
	\end{align*}
	for every $z, w \in K$.
\end{lem}
\begin{proof}
	Note that the sought inequality can be rephrased as positivity of the linear functional
	\begin{align*}
		\varphi \mapsto C_{K} \frac{\Im(\varphi(w))}{\Im(w)} - \frac{\Im(\varphi(z))}{\Im(z)}.
	\end{align*}
	By theorem \ref{pick_functionals} it suffices to prove that there exists a constant $C_{K}$ such that the previous inequality holds for any extreme Pick function, i.e. we should have
	\begin{align*}
		\frac{1}{|\lambda - z|^2} \leq \frac{C_{K}}{|\lambda - w|^2}
	\end{align*}
	for every $\lambda \in \R$. But since
	\begin{align*}
		\left|\frac{\lambda - w}{\lambda - z}\right|^2 \leq \left|1 + \frac{z - w}{\lambda - z}\right|^2 \leq \left(1 + \frac{|z - w|}{\Im(z)}\right)^2
	\end{align*}
	we may take
	\begin{align*}
		C_{K} := \left(1 + \frac{\diam(K)}{\dist(\R, K)}\right)^2
	\end{align*}
\end{proof}

Similarly, one can prove that weakly Pick functions are continuous.

\begin{lause}\label{pick_continuity_lemma}
	Let $\varphi$ be a weakly Pick function. Then $\varphi$ is continuous.
\end{lause}
\begin{proof}
	Our aim is to bound the divided difference $|[z, w]_{\varphi}|$. Now the problem is that this expression is not linear in the function anymore. There's a way to fix this problem however: we bound $\Re(\omega [z, w]_{\varphi})$ for $\omega \in \unitcircle$. This expression is linear in the function, and we have
	\begin{align*}
		|z| \leq C \Leftrightarrow \Re(\omega z) \leq C \text{ for every $\omega \in \unitcircle$}.
	\end{align*}
	Observe that
	\begin{align*}
		\Re\left(\frac{\omega}{(\lambda - z)(\lambda - w)} \right) \leq \frac{1}{|\lambda - z||\lambda - w|} \leq \frac{1}{2} \left( \frac{1}{|\lambda - z|^2} + \frac{1}{|\lambda - w|^2}\right)
	\end{align*}
	for every $\omega \in \unitcircle$. It follows that
	\begin{align*}
		|[z, w]_{\varphi}| \leq \frac{1}{2} \left(\frac{\Im(\varphi(z))}{\Im(z)} + \frac{\Im(\varphi(w))}{\Im(w)}\right)
	\end{align*}
	for any weakly Pick function. Combining this with the Harnack inequality \ref{pick_harnack_lemma} implies that any weakly Pick function is locally Lipschitz, so in particular continuous.
\end{proof}

The previous argument can be easily extended to higher order divided differences.

\begin{lause}\label{Hindmarsh_theorem}
	Let $\varphi$ be a weakly Pick function. Then for any $n \geq 1$ and $z_{0}, z_{1}, \ldots, z_{n}$ we have
	\begin{align*}
		|[z_{0}, z_{1}, \ldots, z_{n}]_{\varphi}| \leq \frac{1}{\Im(z_{2}) \Im(z_{3}) \ldots \Im(z_{n})} \frac{1}{2}\left(\frac{\Im(\varphi(z_{0}))}{\Im(z_{0})} + \frac{\Im(\varphi(z_{1}))}{\Im(z_{1})}\right).
	\end{align*}
	In particular any weakly Pick function is analytic and hence a Pick function.
\end{lause}
\begin{proof}
	Simply note that
	\begin{align*}
		\Re\left(\frac{\omega}{(\lambda - z_{0})(\lambda - z_{1}) \cdots (\lambda - z_{n})} \right) \leq \frac{1}{\Im(z_{2}) \Im(z_{3}) \ldots \Im(z_{n})} \frac{1}{|\lambda - z_{0}||\lambda - z_{1}|}
	\end{align*}
	and follow the argument in the proof of theorem \ref{pick_continuity_lemma}.
\end{proof}

\begin{kor}\label{pick_weakly_pick}
	$\varphi : \Hp \to \C$ is weakly Pick, if and only if it is Pick function.
\end{kor}

\begin{proof}[Proof of theorem \ref{pick_convergence}]
	t.f.i.f \ref{pick_weakly_pick}.
\end{proof}

It is worthwhile to note that as one really only needs to get bound for order $2$ divided differences in the proof of \ref{Hindmarsh_theorem}, one only needs to keep track of small family of pick functionals, in particular ones with support of at most $3$ points. This observation is known as the Hindmarsh theorem.

\section{Pick-Nevanlinna interpolation theorem}

\subsection{Open version}

There's a remarkable generalization to the theorem \ref{pick_weakly_pick}.

\begin{maar}
	Let $X \subset \Hp$. We say that $\varphi : X \to \C$ is weakly Pick on $X$ if $p^{*}(\varphi) \geq 0$ for any Pick functional $p^{*}$ supported on $X$.
\end{maar}

\begin{lause}[Open Pick-Nevanlinna interpolation theorem]\label{open_pick_interpolation}
	Let $U \subset \Hp$ be open and assume that $\varphi$ is weakly Pick on $U$. Then there exists a unique pick function $\tilde{\varphi}$ such that $\restr{\tilde{\varphi}}{U} = \varphi$.
\end{lause}

Note that the proof of \ref{Hindmarsh_theorem} implies that if $\varphi$ is weakly Pick on an open set, it is analytic. The proof of \ref{open_pick_interpolation} is based on the following observation:

\begin{lem}\label{open_pick_lemma}
	Let $U \subset \Hp$ be open and assume that $\varphi$ is weakly Pick on $U$. Let $z_{\infty} \in U$. Then there exists unique weakly Pick $\tilde{\varphi}$ on $U \cap \D(z_{\infty}, \Im(z_{\infty}))$ such that $\restr{\tilde{\varphi}}{U} = \varphi$.
\end{lem}

\begin{proof}
	Take any sequence $z_{0}, z_{1}, \ldots \in U$ converging to $z_{\infty}$: we claim that the Newton series with nodes $z_{0}, z_{1}, \ldots$ gives the (by analyticity necessarily unique) extension for $\varphi$ to $\D(z_{\infty}, \Im(z_{\infty})) \setminus U$.

	To this end take any Pick functional $p^{*}$ supported on  $\D(z_{\infty}, \Im(z_{\infty})) \cup U$ and apply it to our $\tilde{\varphi}$. The functional correponds to some $r \in R_{\infty}(\D(z_{\infty}, \Im(z_{\infty})) \cup U)$. Now if we replace all the evaluations of $p^{*}$ at $\D(z_{\infty}, \Im(z_{\infty})) \setminus U$ by truncation of Newton series (with $N$ terms), we can interpret the result as a new linear functional, say $p^{*}_{N}$. The corresponding rational function is also changed (say to $r_{N} \in R_{\infty}(U)$): all the terms of the form $(\lambda - w)^{-1}$ for $w_{0} \in \D(z_{\infty}, \Im(z_{\infty})) \setminus U$ are replaced by
	\begin{align*}
		\frac{1}{\lambda - z_{0}} + \frac{(w - z_{0})}{(\lambda - z_{0}) (\lambda - z_{1})} + \ldots + \frac{(w - z_{0})\cdots (w - z_{N - 1})}{(\lambda - z_{0})\cdots (\lambda - z_{N})},
	\end{align*}
	and similarly for conjugate terms. Difference between these rational functions,
	\begin{align*}
		\frac{(w - z_{0})\cdots (w - z_{N - 1}) (w - z_{N})}{(\lambda - w) (\lambda - z_{0}) \cdots (\lambda - z_{N})},
	\end{align*}
	can be easily bounded below by
	\begin{align*}
		-\frac{c}{|\lambda - z_{\infty}|^2} \rho^{N},
	\end{align*}
	for some $\rho < 1$ and $c$ not depending on $N$. But this means that $r_{N}$ can't be too small, as $r$ was non-negative to begin with. Indeed, by summing over all the evaluations of $p^{*}$ at $\D(z_{\infty}, \Im(z_{\infty})) \setminus U$, we see that
	\begin{align*}
		r_{N} \geq -\frac{c'}{|\lambda - z_{\infty}|^2} (\rho')^{N}
	\end{align*}
	for some $\rho' < 1$ and $c' > 0$ (again, not depending on $N$). It follows that
	\begin{align*}
		p^{*}(\tilde{\varphi}) = \lim_{N \to \infty} p^{*}_{N}(\varphi) \geq \lim_{N \to \infty} -c' \frac{\Im(\varphi(z_{\infty}))}{\Im(z_{\infty})} (\rho')^{N} = 0,
	\end{align*}
	hence the claim.
\end{proof}


\begin{proof}[Proof of theorem \ref{open_pick_interpolation}]
	We may assume w.l.o.g that $i \in U$. Consider the sequence $z_{k} = (3/2)^{k - 1} i$ so that we have $z_{k + 1} \in \D(z_{k}, \Im(z_{k}))$ for any $k \geq 1$. Applying lemma \ref{open_pick_lemma} repeatedly we may construct functions $(\varphi_{k})_{k > 0}$ such that
	\begin{itemize}
		\item $\varphi_{k} : U_{k} \to \C$, where $U_{k} := U \cup \D(z_{k}, \Im(z_{k}))$, is weakly Pick.
		\item $\restr{\varphi_{k}}{U_{k'}} = \varphi_{k'}$ for any $k \geq k' \geq 1$.
		\item $\restr{\varphi_{k}}{U} = \varphi_{0} := \varphi$ for any $k \geq 1$.
	\end{itemize}
	Since $\cup_{k \geq 1} U_{k} = \Hp$ and $\varphi_{k}$'s agree on their common domains, they determine a function $\tilde{\varphi} : \Hp \to \C$, which trivially satisfies the conditions of the theorem statement.
\end{proof}

\subsection{General version}

Although open Pick-Nevanlinna interpolation theorem is a tool strong enough for our purposes, one cannot simply talk about it without discussing also its big brother, the general case.

\begin{lause}[Pick-Nevanlinna interpolation theorem]\label{pick_interpolation}
	Let $X \subset \Hp$ be arbitrary and assume that $\varphi$ is weakly Pick on $X$. Then there exists a Pick function $\tilde{\varphi}$ such that $\restr{\tilde{\varphi}}{U} = \varphi$.
\end{lause}

It's easy to see that such extension it not unique in general.

The proof is based on the following result.

\begin{lem}\label{pick_extension_lemma}
	Let $X \subset \Hp$ non-empty and $z_{0} \in \Hp \setminus X$. Assume that $\varphi$ is weakly Pick on $X$. Then $\varphi$ can be extended to $z_{0}$, in such a way that the extension is weakly Pick $X \cup \{z_{0}\}$. Moreover, the set of possible values of the extension at $z_{0}$ is a compact subset of $\Hp$.
\end{lem}

Let us first proof the theorem given this lemma.

\begin{proof}[Proof of theorem \ref{pick_interpolation}]
	Consider family of all weakly Pick extensions of $\varphi$ ordered by restriction. It is clear that this family satisfies the conditions of the Zorn's lemma and hence it has a maximal element. But by the previous lemma domain of this maximal element has to have the whole $\Hp$, so it is a sought extension.
\end{proof}

One could avoid the use of Zorn's lemma by, for instance, first extending $\varphi$ to dense subset of $\Hp$ and then noting that this extension continuously extends to the whole of $\Hp$, to a weakly Pick function.

\begin{proof}[Proof of lemma \ref{pick_extension_lemma}]
	Let us first deal with the case of finite $X$. The idea is somewhat similar to the proof of \ref{cheapSpectral}: while in general the extension is very much not unique, if the situation is restricted enough, we are in better situation. Let us denote the sought extension by $\tilde{\varphi}$. Assume first that $\varphi$ is degenerate in $X$ in the sense that $\langle \varphi, r\rangle_{\pickclass} = 0$ for some non-zero $r \in R_{\infty}(X)$.

	Note that since $(\lambda - z_{0})^{-1}$ is bounded on $\R$, by say $M$, we should have
	\begin{align*}
		\left|\langle r \frac{1}{\cdot - z_{0}}, \tilde{\varphi} \rangle_{\pickclass}\right| \leq M \langle r , \varphi \rangle_{\pickclass} = 0,
	\end{align*}
	and hence $\langle r (\cdot - z_{0})^{-1}, \tilde{\varphi} \rangle_{\pickclass} = 0$. Note that then also $\langle r (\cdot - \overline{z_{0}})^{-1}, \tilde{\varphi} \rangle_{\pickclass} = 0$. If $r(z_{0}) \neq 0$, this determines the value $\tilde{\varphi}(z_{0})$. But we may assume that $r$ has only real roots. Indeed, if $r(z_{1}) = 0$ for some $\Im(z_{1}) > 0$, let $\tilde{r} := r \Im(z_{1})^2/N(\cdot - z_{1}) \in R_{\infty}(X)$. As $\tilde{r} \leq r$, also  $\langle \varphi, \tilde{r}\rangle_{\pickclass} = 0$, and $\tilde{r}$ has less non-real roots than $r$.

	Now it remains to be proven that with this choice $\tilde{\varphi}$ is weakly Pick. To this end take any $s \in R_{\infty}(X \cup \{z_{0}\})$. Note that for any $a \in \C$ and $\lambda \in \R$ we have
	\begin{align*}
		\tilde{s}(\lambda) := s(\lambda) + \left(\frac{a}{\lambda - z_{0}} + \frac{\overline{a}}{\lambda - \overline{z_{0}}}\right) r(\lambda) + 2 |a| M r(\lambda) \geq 0.
	\end{align*}
	By picking $a$ suitably, $\tilde{s}$ doesn't have pole at $z_{0}$ (or $\overline{z_{0}}$), and since $\varphi$ is weakly Pick, we thus have
	\begin{align*}
		\langle s, \tilde{\varphi} \rangle_{\pickclass} = \langle \tilde{s}, \varphi \rangle_{\pickclass} \geq 0,
	\end{align*}
	the claim.

	If $\varphi$ is not-degenerate, we may certainly find $c_{0} > 0$ such that $\varphi_{c_{0}} := \varphi - c_{0} i$ is weakly Pick on $X$ and degenerate, and if we find extension for $\varphi_{c_{0}}$, we get one also for $\varphi$.

	The proof of \ref{pick_continuity_lemma} immediately implies that the set of suitable values $\tilde{\varphi}(z_{0})$ bounded, and as it is also clearly closed, it is compact.

	Let us now move to the case of general non-empty $X$. For any finite subset $F \subset X$ denote the set of possible values of a weakly Pick extension of $\restr{\varphi}{F}$ at $z_{0}$ by $W_{F}$. We clearly have $W_{F_{1} \cup F_{2}} \subset W_{F_{1}} \cap W_{F_{2}}$ and hence
	\begin{align*}
		\{ W_{F} | \text{$F$ is a finite subset of $X$}\}
	\end{align*}
	is a family of compact sets with finite intersection property. Consequently, their intersection is non-empty and compact.
\end{proof}

\section{Notes and references}

Pick functions and representation theorem \ref{pick_nevanlinna_herglotz_representation_theorem} are discussed in numerous sources; see for example (?). Pick-Nevanlinna interpolation theorem \ref{pick_interpolation} was first observed and proved independently by Pick \cite{Pick} and Nevanlinna \cite{Nevan}. Since then, many different approaches exist, see (?) for a survey. Approach taken in text is similar to (?). \ref{open_pick_interpolation} first appeared in \cite{Hind} (formulated in terms of Pick matrices).

\begin{comment}

TODO:
\begin{itemize}
	\item Examples of representing measures behind functions and functions behind representing measures
	\item Spectral commutant lifting theorem
	\item Use Morera's theorem to prove weak Hindmarsh's theorem
	\item Nice formula for finite Pick extension (rational function case)
\end{itemize}

\end{comment}






