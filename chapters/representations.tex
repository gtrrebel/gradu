\chapter{Representations}

Over the course of this thesis we have mentioned various representations results of the following form:

\[
f(x) = \int h_{t}(x) d \mu (t),
\]
or
\[
f = \int h_{t} d \mu(t).
\]
Here $\mu$ is Borel measure on some set and $f$ and $h_{t}$'s are functions of some kind. Functions $h_{t}$ should be thought of some kind of basis functions. Although such results have been hardly used, one cannot just leave them unmentioned.

Most of the representation results in the thesis can be understood in terms of Choquet theory. The idea is the following: the sets we are concerned with are convex and the functions $h_{t}$ the extreme points of these convex set. Extreme points are the points that can't be expressed as a non-trivial linear combination of points in the set.

Now if one has say compact convex set in $K \subset \R^{n}$, $K$ should be roughly given by its boundary: compact convex sets are equal if and only they have same boundary. But more is true: every point in $K$ can be expressed as a convex combination of extreme points of $K$ (actually of at most $n + 1$ extreme points).

Same holds true much more generally, in infinite dimensional spaces.