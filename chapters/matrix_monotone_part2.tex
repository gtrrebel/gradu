\chapter{Matrix monotone functions -- part 2}

\section{Main theorem}

Let's come back to theorem \ref{heuristic_loewner}. In the proof we noticed that if $f$ is $n$-monotone and analytic, then the integrals
\begin{align*}
	\frac{1}{2 \pi i} \int_{\gamma}  \frac{q(z) \overline{q(\overline{z})}}{z^{2 n}} f(z) dz
\end{align*}
should be all positive. This integral evaluates to $(2 n - 1)$'th derivative of the function $f(z) q(z) \overline{q(\overline{z})}$. Given polynomial $q$, we denote by $N(q)$ polynomial with
\begin{align*}
	N(q)(z) = q(z) \overline{q(\overline{z})}.
\end{align*}
With the $k$-tone language theorem rewrites to

\begin{lause}\label{main_theorem}
	$f \in P_{n}(a, b)$, if and only if $f N(q)$ is $(2 n - 1)$-tone for every (complex) polynomial $q$ of degree less than $n$.
\end{lause}

This statement makes sense without regularity issues and it is true. There's also many different ways to talk about the polynomial part.
\begin{lem}\label{polynomial_lemma}
	Let $h : \C \to \C$ and $n \geq 1$. Then the following are equivalent.
	\begin{enumerate}[(i)]
		\item $h$ is polynomial non-negative on real line and of degree less than $2 n$.
		\item There exists complex polynomial of degree less than $n$ such that $h = N(q)$.
		\item There exists two real polynomials $q_{1}$ and $q_{2}$ of degree less than $n$ such that $h = q_{1}^2 + q_{2}^2$.
	\end{enumerate}
\end{lem}
\begin{proof}
	$(i) \Rightarrow (ii)$: If $h$ is non-negative on real axis, it's roots all appear in pairs (which there are less than $n$): either with strict complex conjugate pairs, of pairs of double real roots. We may take $q$ to be $\sqrt{a_{n}}\prod (z - z_{i})$ where $z_{i}$ range over representatives of all the pairs and $a_{n}$ is the leading coefficient of $h$.

	$(ii) \Rightarrow (iii)$: If $q$ has single conjugate pair $(z_{0}, \overline{z_{0}})$ of roots we have
	\begin{align*}
		(z - z_{0}) (z - \overline{z_{0}}) = z^2 - 2 \Re(z_{0}) + |z_{0}|^2 = (z - \Re(z_{0}))^2 + \Im(z_{0})^2,
	\end{align*}
	so we may take $q_{1} = \cdot - \Re(z_{0})$ and $q_{2} = \Im(z_{0})$. But if $ N(q) = q_{1}^2 + q_{2}^2$ and $ N(r) = r_{1}^2 + r_{2}^2$, then
	\begin{align*}
		N(q r) = N(q) N(r) = (q_{1} r_{1} + q_{2} r_{2})^2 + (q_{1} r_{2} - q_{2} r_{1})^2,
	\end{align*}
	so polynomial of higher order can be dealt with inductively.

	$(iii) \Rightarrow (i)$: This is clear.
\end{proof}

Theorem \ref{main_theorem} gives nice resolution for the regularity issues of its predecessor, theorem \ref{heuristic_loewner}. Recall that $(2 n - 1)$-tone functions are $C^{2 n - 3}$ and their $(2 n - 3)$'th derivative is convex. As convex functions are are twice differentiable almost everywhere, the Dobsch matrix makes sense almost everywhere, so the condition can be understood in the ``almost everywhere" -sense.

Aim of this chapter is to give a honest proof for \ref{main_theorem}: we aim to prove that given $n \geq 1$, open interval $(a, b)$ and $f : (a, b) \to \R$, we have 
\begin{align*}
	& f \in P_{n}(a, b) \\
	\Leftrightarrow &[x_{0}, x_{1}, x_{2}, \ldots, x_{2 n - 1}]_{f N(q)} \text{ for any $q \in \C_{n - 1}[x]$ and $a < x_{0} < \ldots < x_{2 n - 1} < b$}.
\end{align*}
Note that we proof of \ref{heuristic_loewner} can be interpreted as saying
\begin{align*}
	& D_{n} f_{A}(H) \geq 0 \text{ for any $A \in \H^{n}_{(a, b)}$ and $H \geq 0$ with $\rank(H) = 1$} \\
	\Leftrightarrow &[\lambda_{1}, \lambda_{1}, \ldots, \lambda_{2 n}, \lambda_{2 n}]_{f N(q)} \text{ for any $q \in \C_{n - 1}[x]$ and $a < \lambda_{1} < \ldots < \lambda_{n} < b$}.
\end{align*}
The idea of the proof was to note that the quadratic form of the derivative rewrites to such divided difference, and $\lambda$'s are the eigenvalues of $A$.

To avoid regularity issues we simply do the same thing without limits.

\begin{maar}
	Let us call pair $(A, B) \in \H(V)^{2}$ a \textbf{projection pair} if $B - A = v v^{*}$ for some $v \in V$. Note that such $v$ is always unique up to phase. Let us say that a projection pair $(A, B)$ is \textbf{strict}, if whenever $B - A = v v^{*}$ then $v$ is not orthogonal to any eigenvector of $A$. 
\end{maar}

\begin{lem}\label{main_lemma}
	If $a < \lambda_{1} < \lambda_{2} < \ldots < \lambda_{2 n - 1} < \lambda_{2 n} < b$ and $q \in \C_{n - 1}[x]$, we may find a strict projection pair $(A, B)$ such that
	\begin{align*}
		\langle (f(B) - f(A)) w, w \rangle = [\lambda_{1}, \lambda_{2}, \lambda_{3}, \lambda_{4}, \ldots, \lambda_{2n - 1}, \lambda_{2 n}]_{f N(q)}
	\end{align*}
	for any $f : (a, b) \to \R$.

	Conversely, if $(A, B)$ is a strict projection pair and $w \in V$, then there exists $a < \lambda_{1} < \lambda_{2} < \ldots < \lambda_{2 n - 1} < \lambda_{2 n} < b$ and polynomial $q \in \C_{n - 1}[x]$, such that for any $f : (a, b) \to \R$ we have
	\begin{align*}
		\langle (f(B) - f(A)) w, w \rangle = [\lambda_{1}, \lambda_{2}, \lambda_{3}, \lambda_{4}, \ldots, \lambda_{2n - 1}, \lambda_{2 n}]_{f N(q)}.
	\end{align*}
\end{lem}

Before trying to understanding the lemma we use it to prove theorem \ref{main_theorem}.

\begin{proof}
	Assume first that $f \in P_{n}(a, b)$. We need to prove that $f q \overline{q(\overline{\cdot})}$ is $(2 n - 1)$-tone for any $q \in \C_{n - 1}[x]$ But any divided difference of such $f q \overline{q(\overline{\cdot})}$ can be expressed by the main lemma \ref{main_lemma} as $\langle (f(B) - f(A)) w, w \rangle$ for some projection pair $(A, B)$, and the previous is non-negative by the assumption.

	Conversely, assume that $f h$ is $(2 n - 1)$-tone for any suitable $h$ and take any $A \leq B$. Write $B - A = \sum_{i = 1}^{n} c_{i} P_{v_{i}}$ for some $c_{i} \geq 0$. To prove that $f(B) - f(A) \geq 0$ we simply need to prove that $f(A + \sum_{i = 1}^{k} c_{i} P_{v_{i}}) - f(A + \sum_{i = 1}^{k - 1} c_{i} P_{v_{i}}) \geq 0$ for any $1 \leq k \leq n$, as $f(B) - f(A)$ is sum of such terms. We may hence assume that $(A, B)$ projection pair.

	We may also assume that $(A, B)$ is strict. Indeed, if this would not be the case, we could decompose $V = \vspan\{v_{1}\} \oplus V'$, where $v_{1}$ is the eigenvector, and factorize $A = \arestr{A}{\vspan\{v_{1}\}} \oplus \arestr{A}{V'}$ and $P_{w} = 0 \oplus \arestr{(P_{w})}{V'}$. But now checking that $f(B) - f(A) \geq 0$ boils down to checking that $f(\arestr{B}{V'}) - f(\arestr{A}{V'}) \geq 0$, which would follow if we could prove that $f \in P_{n - 1}(a, b)$. But this follows if we add the sentence ``We induct on $n$." as the first sentence of this proof.

	Finally in this case, by the lemma \ref{main_lemma} we may find $a < \lambda_{1} < \lambda_{2} < \ldots < \lambda_{2 n - 1} < \lambda_{2 n} < b$ and $q \in \C_{n - 1}[x]$ such that
	\begin{align*}
		\langle (f(B) - f(A)) w, w \rangle = [\lambda_{1}, \lambda_{2}, \lambda_{3}, \lambda_{4}, \ldots, \lambda_{2n - 1}, \lambda_{2 n}]_{f N(q)} \geq 0
	\end{align*}
	and we are finally done.

	In the ``if"-direction we could alternatively make use of the continuity of $f$, which is guaranteed by the lemma \ref{k-tone_smooth}

\end{proof}

\section{Main lemma}

It remains to understand what is going on with lemma \ref{main_lemma}. The surprising part about it is the following fact about eigenvalues.

\begin{lem}\label{projection_eigenvalues}
	Let $(A, B)$ be a projection pair. Then
	\begin{align*}
		\lambda_{1}(B) \geq \lambda_{1}(A) \geq \lambda_{2}(B) \geq \lambda_{2}(A) \geq \ldots \geq \lambda_{n}(B) \geq \lambda_{n}(A).
	\end{align*}
	$(A, B)$ is strict if and only if all the inequalities are strict.

	Conversely, if we are given any two interlacing sequences $b_{1} \geq a_{1} \geq b_{2} \geq a_{2} \geq \ldots \geq b_{n} \geq a_{n}$ we may find a projection pair $(A, B)$ with $\spec(A) = \{a_{i}\}_{i = 1}^{n}$ and $\spec(B) = \{b_{i}\}_{i = 1}^{n}$.
\end{lem}

This proposition is based on the following explicit relationship between characteristic polynomials of a projection pair.

\begin{lem}\label{projection_characteristic_polynomial}
	Let $A, B \in \H$ be a projection pair. Then
	\begin{align*}
		\det(B - z I) = \det(A - z I) \left(1 + \langle (A - z I)^{-1}v, v\rangle\right).
	\end{align*}
\end{lem}
\begin{proof}
	Write the the matrices $A$ and $B$ in the basis where the first vector is parallel to $v$. Now the matrices only differ at the upper-left corner, where the difference is $\|v\|^2$. Expanding the determinant this implies that
	\begin{align*}
		\det(B - z I) =& \det(A - z I) \\
		&+ \|v\|^2 \left(\text{determinant of $A - zI$ with first row and column removed} \right).
	\end{align*}
	However, by the Cramer rule the determinant equals the upper-left corner of the matrix of $(A - z I)^{-1}$, i.e. $\det(A - z I) \langle (A - zI)^{-1} v, v \rangle/\|v\|^2$. Combining these observations yields the claim.
\end{proof}

\begin{proof}[Proof of lemma \ref{projection_eigenvalues}]
	Note that if $v$ is orthogonal to one of the eigenvectors of $A$, $P_{v}$ doesn't affect this eigenspace, so we may forget it and restrict our attention to a smaller space. Similarly for the converse: if $a_{i} = b_{j}$ for some $1 \leq i, j \leq n$ we can forget $a_{i}$ and $b_{j}$, and solve the remaingn problem on smaller space. We may hence assume that the pair $(A, B)$ is strict and the numbers the inequalities in the converse are strict.

	Consider the function
	\begin{align*}
		z \mapsto 1 + \sum_{i = 1}^{n} \frac{|\langle v, e_{i} \rangle|^2}{a_{i} - z}.
	\end{align*}
	It has $n$ poles of negative residue so it has a root between any two poles. Also it tends to $1$ at infinity so it has also root on $(a_{1}, \infty)$. Hence it has $n$ roots. All these roots are eigenvalues of $B$ so they are exactly the eigenvalues. This implies one direction.

	For the converse take first $A$ with the given eigenvalues. By the previous lemma we now just want to choose $v$ in such a way that
	\begin{align*}
		\frac{p_{B}(z)}{p_{A}(z)} = 1 + \langle (A - z I)^{-1}v, v\rangle= 1 + \sum_{i = 1}^{n} \frac{|\langle v, e_{i} \rangle|^2}{a_{i} - z},
	\end{align*}
	But this is clearly achieveable if can show that the residues of $p_{B}(z)/p_{A}(z)$ are negative, which follows easily from the interlacing property. Hence the converse.
\end{proof}

Let us then complete the proof of theorem \ref{main_theorem} by proving the lemma \ref{main_lemma}.

\begin{proof}[Proof of lemma \ref{main_lemma}]
	The proof is based on lemmas \ref{projection_eigenvalues} and \ref{projection_eigenvalues}. To find the connection we first assume $f$ is entire. Then if and $(A, B)$ is a strict projetion pair with $B - A = v v^{*}$ for some $v \in V$ and $w \in V$ we have
	\begin{align*}
		&= \langle (f(B) - f(A)) w, w \rangle \\
		&= \frac{1}{2 \pi i}\int_{\gamma} \langle (z I - B)^{-1} v, w \rangle  \langle (z I - A)^{-1} w, v \rangle f(z) dz \\
		&= \frac{1}{2 \pi i}\int_{\gamma} \frac{\det(z I - A)\langle (z I - B)^{-1} v, w \rangle \det(z I - B) \langle (z I - A)^{-1} w, v \rangle}{\det(z I - A) \det(z I - B)} f(z) dz.
	\end{align*}
	The integrand equals
	\begin{align*}
		\frac{h(z)}{\prod_{i = 1}^{n}(z - \lambda_{i}(A)) \prod_{i = 1}^{n}(z - \lambda_{i}(B))} f(z),
	\end{align*}
	where $h(z) = \det(z I - B)\langle (z I - B)^{-1} v, w \rangle \det(z I - A) \langle (z I - A)^{-1} w, v \rangle$.
	\begin{lem}
		If $(A, B)$ is a projection pair with $B - A = v v^{*}$ then
		\begin{align*}
			\det(z I - A) (z I - A)^{-1} v = \det(z I - B) (z I - B)^{-1} v
		\end{align*} 
	\end{lem}
	\begin{proof}
		As $z I - A = z I - B + v v^{*}$, multiplying both sides from left by $(z I - A)$ leads to the equivalent
		\begin{align*}
			\det(z I - A) v = \det(z I - B) (1 + \langle (z I - B)^{-1} v, v \rangle) v
		\end{align*}
		which follows from \ref{projection_characteristic_polynomial}.
	\end{proof}
	By the previous lemma we have $\det(z I - B) \langle (z I - B)^{-1} w, v \rangle = \overline{\det(\overline{z} I - A) \langle (\overline{z} I - A)^{-1} w, v \rangle}$ so
	if we write $q(z) = \det(z I - A) \langle (z I - A)^{-1} w, v \rangle$, we have
	\begin{align*}
		\langle (f(B) - f(A)) w, w \rangle = [\lambda_{1}(A), \ldots, \lambda_{n}(A), \lambda_{1}(B), \ldots, \lambda_{n}(B)]_{f N(q)}.
	\end{align*}
	Note that this identity evidently holds without any extra smootness assumptions.

	Now when $(A, B)$ ranges over all strict projection pairs, the permutations of tuples
	\begin{align}
	(\lambda_{1}(A), \ldots, \lambda_{n}(A), \lambda_{1}(B), \ldots, \lambda_{n}(B))
	\end{align}
	range over all tuples of distinct numbers on $(a, b)$. Hence to prove the lemma, we should prove that for fixed strict projection pair $(A, B)$, as $w$ ranges over $V$, $q$ ranges over $\C_{n - 1}[x]$. But this is clear as components of $\det(z I - A)(z I - A)^{-1} v$ with respect to eigenbasis of $A$, $(e_{i})_{i = 1}^{n}$ are $p_{j}(z) = \prod_{i \neq j}(z - \lambda_{i}(B)) \langle v, e_{i} \rangle$, which are clearly linearly independent polynomials over $\C$.

	To recap, the map
	\begin{align*}
		V &\to \C_{n - 1}[x] = \{\text{Complex polynomials of degree at most } (n - 1) \} \\
		w &\mapsto \det(z I - A) \langle (z I - A)^{-1} v, w \rangle
	\end{align*}
	is antilinear bijection, the correspondece between $w$ and $q$.
\end{proof}

\section{Dual pairing}

Proof of theorem \ref{main_theorem} is actually missing one more detail we need in the induction step.

\begin{lem}\label{k_tone_cor}
	If $n \geq 1$ and $f N(q)$ is $(2 n + 1)$:tone for $h \in \C_{n}[x]$, then $f N(r)$ is $(2 n - 1)$-tone for every $r \in \C_{n - 1}[x]$.
\end{lem}

To understand this result, recall that for analytic $f$ and $a < x_{0} < x_{1} < \ldots < x_{2 n - 1} < b$ we have
\begin{align*}
	[x_{0}, x_{1}, \ldots, x_{2 n - 1}]_{f N(q)} = \frac{1}{2 \pi i} \int_{\gamma} \frac{N(q)}{(z - z_{0}) (z - z_{1}) \cdots (z - z_{2 n - 1})} f(z) dz,
\end{align*}
for suitable $\gamma$. One way to interpret this identity is to consider it as a linear map: for given analytic $f$ we have the map
\begin{align*}
	r \mapsto  \frac{1}{2 \pi i} \int_{\gamma} r(z) f(z) dz,
\end{align*}
where $r$ is a rational functions with its poles in the domain of $f$. Note that all this makes formally sense for arbitrary $f$ given that the poles of $r$ are simple. This motivates us to define a dual pairing $\langle \cdot, \cdot \rangle_{L}$ (over $\R$) between $\R^{(a, b)}$ and rational functions with simple poles on $(a, b)$, for which
\begin{align*}
	\langle f, r \rangle_{L} = \frac{1}{2 \pi i} \int_{\gamma} r(z) f(z) dz,
\end{align*}
for analytic $f$. We could of course replace $(a, b)$ by any subset of $\C$.

Now theorem \ref{main_theorem} is just saying that $f \in P_{n}(a, b)$ if $\langle f, r \rangle_{L}$ for $r$ of the form
\begin{align*}
	r(z) = \frac{N(q)}{(z - z_{0}) (z - z_{1}) \cdots (z - z_{2 n - 1})}
\end{align*}
where $a < z_{0} < \ldots < z_{2 n - 1} < b$ and $q \in \C_{n - 1}[x]$. Let us denote this family of rational functions by $R_{n}(a, b)$. Now, in order to prove the lemma we should prove that we have
\begin{align*}
	\cone(R_{n}(a, b)) \subset \cone(R_{n + 1}(a, b))
\end{align*}

\begin{proof}[Proof lemma \ref{k_tone_cor}]
	Take any $r \in R_{n}(a, b)$. If there was no condition on the order of the poles, we could simply note that
	\begin{align*}
		r = r \frac{N(z - c)}{(z - c) (z - c)} \in R_{n + 1}(a, b).
	\end{align*}
	Even though that doesn't quite work, we can modify the trick a little: we have
	\begin{align*}
		r = \frac{1}{2} r \frac{N((z - c))}{(z - c) (z - \frac{c + d}{2})} +  \frac{1}{2} r\frac{N((z - d))}{(z - d) (z - \frac{c + d}{2})} \in \cone(R_{n + 1}(a, b)).
	\end{align*}
	Now simply choose $c, d \in (a, b)$ so that the poles remain simple.
\end{proof}

\section{Section title}

Let's move our focus to classes $P_{\infty}(a, b)$. Using the earlies ideas we rewrite theorem \ref{heuristic_loewner} in the following form.

\begin{lause}\label{better_loewner}
	$f \in P_{\infty}(a, b)$, if and only if $f$ is analytic and for every $n \geq 1$ and $q \in \C_{n - 1}[x]$ the function $f N(q)$ is $(2 n - 1)$-tone.
\end{lause}

Without the analyticity condition this would immediately follow from \ref{main_theorem}, and the statement would make perfect sense without it. It is nevertheless true that the functions in class $P_{\infty}(a, b)$ are analytic. One could use Bernstein type arguments and tricks (see proof of the theorem \ref{bernstein_theorem}) to convince oneself that this indeed the case, but actually there's a lot more going on.

First of all, the dual pairing thinking leads much more satisfactory conclusion in the case $n = \infty$.

\begin{lem}
	We have
	\begin{align*}
		&R_{\infty}(a, b) := \bigcup_{n = 1}^{\infty} R_{n}(a, b) = \cone\left( \bigcup_{n = 1}^{\infty} R_{n}(a, b)\right) \\
		= &\{ \text{rational functions with simple poles on $(a, b)$ non-negative} \\
		& \text{ on $\R \setminus (a, b)$ and decay $r(z) = O(|z|^{-2})$ at $\infty$} \}.
	\end{align*}
\end{lem}
\begin{proof}
	It is sufficient to prove that if $r$ is rational function with simple poles on $(a, b)$, which is non-negative on $\R \setminus (a, b)$ and has decay $r(z) = O(|z|^{-2})$ at $\infty$, then $r \in R_{n}(a, b)$ for some $n \geq 1$. So pick such $r$.

	Let $a < x_{0} < x_{1} < \ldots < x_{2 n - 1} < b$ be the points where $r$ changes its sign. Now $p := r \prod_{i = 0}^{2 n - 1} (\cdot - x_{i})$ is polynomial of degree less than $2 n$ non-negative on $\R$, so it is of the form $N(q)$ for some $q \in \C_{n - 1}[x]$. Hence we have
	\begin{align*}
		r = \frac{N(q)}{p} \in R_{n}(a, b).
	\end{align*}
\end{proof}

There's one little problem with the previous proof: $r$ might change its sign at $a$ or at $b$ in which case the argument fails. So we actually proved
\begin{align*}
	&R_{\infty}(a, b) = \{ \text{rational functions with simple poles on $(a, b)$ non-negative} \\
	& \text{ on $\R \setminus [c, d]$ for some $a < c < d < b$, and decay $r(z) = O(|z|^{-2})$ at $\infty$} \}.
\end{align*}

\section{Notes and references}
