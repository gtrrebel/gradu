\chapter{Matrix $k$-tone functions}

\section{Matrix convex functions}

Having charaterized matrix monotone functions it is natural to ask what happens with convex functions.

\begin{maar}
	We say that $f : (a, b) \to \R$ is matrix convex of order $n$ if for every $A, B \in \H^{n}(V)$ and $t \in [0, 1]$ we have
	\begin{align*}
		f(t A + (1 - 1) B) \leq t f(A) + (1 - t) f(B).
	\end{align*}
\end{maar}

We denote class of matrix convex functions of order $n$ on interval $(a, b)$ by $P^{2}_{n}(a, b)$. Many of the properties of matrix monotone functions translate immediately to matrix convexity.

\begin{prop}
	Let $(a, b) \subset \R$ be an open interval $n \geq 1$. Then
	\begin{enumerate}
		\item $P_{n}^{2}(a, b)$ is a convex closed cone.
		\item $P_{n + 1}^{2}(a, b) \subset P_{n}^{2}(a, b)$.
		\item $\left(x \mapsto \alpha_{2} x^{2} + \alpha_{1} x + \alpha_{0}\right) \in P^{2}_{n}(a, b)$ if $\alpha_{n} \geq 0$.
		\item $\left(x \mapsto |x|^{-1}\right) \in P^{2}_{n}(a, b)$ whenever $0 \notin (a, b)$.
	\end{enumerate}
\end{prop}
\begin{proof}
	\begin{enumerate}
		\item This is clear.
		\item The proof is essentially the same as in the monotone case.
		\item We have
		\begin{align*}
			& t (\alpha_{2} A^2 + \alpha_{1} A + \alpha_{0} I) + (1 - t) (\alpha_{2} B^2 + \alpha_{1} B + \alpha_{0} I) \\
			&- (\alpha_{2} (t A + (1 - t) B)^2 + \alpha_{1} (t A + (1 - t) B) + \alpha_{0} I) \\
			=& \alpha_{2} t (1 - t) (A - B)^{2}.
		\end{align*}
		\item It's is clearly sufficient to prove that $x \mapsto x^{-1}$ is convex on $(0, \infty)$. For this we should prove that for any $A, B > 0$ and $t \in [0, 1]$ we have
		\begin{align*}
			t A^{-1} + (1 - t) B^{-1} \geq (t A + (1 - t) B)^{-1}.
		\end{align*}
		By doing the congruence trick, i.e. $*$-conjugating by $A^{\frac{1}{2}}$ and setting $X = A^{-\frac{1}{2}} B A^{-\frac{1}{2}}$ we are left with
		\begin{align*}
			t + (1 - t) X^{-1} \geq (t + (1 - t) X)^{-1},
		\end{align*}
		which would follow if we can prove the respective scalar inequality. But
		\begin{align*}
			t + (1 - t) x^{-1} - (t + (1 - t) x)^{-1} = \frac{t (1 - t)}{x (t + (1 - t) x)} (x - 1)^{2},
		\end{align*}
		so the scalar inequality is true.
	\end{enumerate}
\end{proof}

With linear combinations of the previous examples we can again build large number of $n$-convex functions. TODO

By now it ought to be no surprise that not every convex function is matrix convex. Canonical counterexample is absolute value. In turns out that we have

\begin{prop}
	Let $v, w \in V \setminus \{0\}$. Then
	\begin{align*}
	P_{v} + P_{w} \geq |P_{v} - P_{w}|,
	\end{align*}
	if and only if $v$ and $w$ are parallel or orthogonal, i.e. if and only if $P_{v}$ and $P_{w}$ commute.
\end{prop}
\begin{proof}
	As everything is happening in (at most) two dimensional subspace of $V$, we may assume that $V$ is two dimensinal in the first place. Note that in this case $P_{v} - P_{w}$ has $0$ trace, so its eigenvalues are additive inverses of each other. Consequently the absolute value is multiple of identity.

	It follows that both sides of the inequality commute, and as they are both positive, it suffices to check when the squared inequality holds. This leads to an equivalent inequality
	\begin{align*}
		P_{v} P_{w} + P_{w} P_{v} \geq 0,
	\end{align*}
	which was already discussed in \ref{symmetric_projection}.
\end{proof}

\section{Convexity and the second derivative}

So far everything has worked pretty much the same way as with the monotone functions, but with derivative things to start to look very different. Theorem \ref{monotone_derivative} has natural analog in the convex case.

\begin{lause}
	Let $n \geq 1$ and $f \in C^{2}(a, b)$.
	Then the following are equivalent:
	\begin{enumerate}[(i)]
	\item $f \in P^{2}_{n}(a, b)$.
	\item For any $A, H \in \H^{n}_{(a, b)}$ we have
	\[
		D^{2}_{n}f_{A}(H) \geq 0.
	\]
	\item For any $A, H \in \H^{n}_{(a, b)}$ and $v \in V$ the map
	\[
		t \mapsto \langle f(A + t H) v, v \rangle
	\]
	is convex.
	\end{enumerate}
\end{lause}

Now, there is a big problem here: there are only three conditions in this theorem. In the monotone case we could take $H$ to be projection, but with the convex case this is not (a priori) possible anymore. Recall that working with projections was rather easy (we could even have very explicit formulas for everything) but with general positive maps similar arguments are hopeless. TODO

TODO

\section{Matrix $k$-tone functions}

After having defined the notion of $k$-tone function in the real setting, it is natural to ponder what happens with matrix setting. Defining the notion itself is already a bit cumbersome: with monotone and convex functions the usual definitions make immediately sense but divided differences cause some problems. One cannot simply say that $f$ is matrix monotone if
\begin{align*}
	[A, B]_{f} = \frac{f(B) - f(A)}{B - A},
\end{align*}
since the right-hand side doesn't make much sense. We can however use an equivalent definition from the theorem \ref{monotone_derivative}.

\section{Basic properties}

\begin{maar}
	We say that $f : (a, b) \to \R$ is matrix $k$-tone of order $n$ if for every $A \in \H^{n}(V)$ and $B \in \H^{n}_{+}(V)$ and $v \in V$ the function
	\begin{align*}
		t \mapsto \langle f(A + t B) v, v \rangle
	\end{align*}
	is $k$-tone.
\end{maar}

Denote the class of matrix $k$-tone functions of order $n$ on interval $(a, b)$ as $P^{k}_{n}(a, b)$ (so $P^{1}_{n}(a, b) = P_{n}(a, b)$).

This definition doesn't exactly coincide with our definition for matrix convex functions, where we needed no assumption on the ``sign" of $B$. As we will later see, however, this alternate definition leads to same set of functions.


As in the monotone case, we can list many natural properties of classes $P^{k}_{n}(a, b)$, proofs of which are very similar to the monotone case.

\begin{prop}
	Let $(a, b) \subset \R$ be an open interval $n \geq 1$, and $k \geq 1$. Now
	\begin{enumerate}
		\item $P_{n}^{k}(a, b)$ is a convex closed cone.
		\item $P_{n + 1}^{k}(a, b) \subset P_{n}^{k}(a, b)$.
		\item $\left(x \mapsto \alpha_{k} x^{k} + \ldots + \alpha_{1} x + \alpha_{0}\right) \in P^{k}_{n}(a, b)$ if $\alpha_{n} \geq 0$.
		\item $\left(x \mapsto (-1)^k x^{-1}\right) \in P^{k}_{n}(a, b)$.
	\end{enumerate}
\end{prop}
\begin{proof}
	TODO
\end{proof}

Not surprisingly, we have also the following derivative characterization.

\begin{lause}
	Let $n, k \geq 1$ and $f \in C^{k}(a, b)$.
	Then the following are equivalent:
	\begin{enumerate}[(i)]
	\item $f \in P^{k}_{n}(a, b)$.
	\item For any $A \in \H^{n}_{(a, b)}$ and $H \geq 0$ we have
	\[
		D^{k}_{n}f_{A}(H) \geq 0.
	\]
	\end{enumerate}
\end{lause}
\begin{proof}
	TODO
\end{proof}

But there's a problem: there's no obvious way to change the $H$ in the statement of the previous theorem to one-dimensional projection, when $k > 1$. The issue is that when $k > 1$, the map
\begin{align*}
	H \mapsto D^{k}_{n}f_{A}(H)
\end{align*}
is not linear anymore! It's a horrible mess instead.

TODO

\begin{itemize}
	\item Is this section really needed?
	\item How to deal with smoothness issues cleanly?
\end{itemize}







