\chapter{Matrix $k$-tone functions}

After having defined the notion of $k$-tone function in the real setting, it is natural to ponder what happens with matrix setting. Defining the notion itself is already a bit cumbersome: with monotone and convex functions the usual definitions make immediately sense but divided differences cause some problems. One cannot simply say that $f$ is matrix monotone if
\begin{align*}
	[A, B]_{f} = \frac{f(B) - f(A)}{B - A},
\end{align*}
since the right-hand side doesn't make much sense. We can however use an equivalent definition from the theorem \ref{monotone_derivative}.

\section{Basic properties}

\begin{maar}
	We say that $f : (a, b) \to \R$ is matrix $k$-tone of order $n$ if for every $A \in \H^{n}(V)$ and $B \in \H^{n}_{+}(V)$ and $v \in V$ the function
	\begin{align*}
		t \mapsto \langle f(A + t B) v, v \rangle
	\end{align*}
	is $k$-tone.
\end{maar}

Denote the class of matrix $k$-tone functions of order $n$ on interval $(a, b)$ as $P^{k}_{n}(a, b)$ (so $P^{1}_{n}(a, b) = P_{n}(a, b)$).

This definition doesn't exactly coincide with our definition for matrix convex functions, where we needed no assumption on the ``sign" of $B$. As we will later see, however, this alternate definition leads to same set of functions.


As in the monotone case, we can list many natural properties of classes $P^{k}_{n}(a, b)$, proofs of which are very similar to the monotone case.

\begin{prop}
	Let $(a, b) \subset \R$ be an open interval $n \geq 1$, and $k \geq 1$. Now
	\begin{enumerate}
		\item $P_{n}^{k}(a, b)$ is a convex closed cone.
		\item $P_{n + 1}^{k}(a, b) \subset P_{n}^{k}(a, b)$.
		\item $\left(x \mapsto \alpha_{k} x^{k} + \ldots + \alpha_{1} x + \alpha_{0}\right) \in P^{k}_{n}(a, b)$ if $\alpha_{n} \geq 0$.
		\item $\left(x \mapsto (-1)^k x^{-1}\right) \in P^{k}_{n}(a, b)$.
	\end{enumerate}
\end{prop}
\begin{proof}
	TODO
\end{proof}

Not surprisingly, we have also the following derivative characterization.

\begin{lause}
	Let $n, k \geq 1$ and $f \in C^{k}(a, b)$.
	Then the following are equivalent:
	\begin{enumerate}[(i)]
	\item $f \in P^{k}_{n}(a, b)$.
	\item For any $A \in \H^{n}_{(a, b)}$ and $H \geq 0$ we have
	\[
		D^{k}_{n}f_{A}(H) \geq 0.
	\]
	\end{enumerate}
\end{lause}
\begin{proof}
	TODO
\end{proof}

But there's a problem: there's no obvious way to change the $H$ in the statement of the previous theorem to one-dimensional projection, when $k > 1$. The issue is that when $k > 1$, the map
\begin{align*}
	H \mapsto D^{k}_{n}f_{A}(H)
\end{align*}
is not linear anymore! It's a horrible mess instead.

\section{TODO}

\begin{itemize}
	\item Is this section really needed?
	\item How to deal with smoothness issues cleanly?
\end{itemize}







