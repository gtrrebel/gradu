\chapter{Introduction}

\section{Foreword}

This master's thesis delves into the theory of matrix monotone functions. Matrix monotonicity is generalization of standard monotonicity f real functions: now we are just having functions mapping matrices to matrices. Formally, $f$ is \textit{matrix monotone} if for any two matrices $A$ and $B$ such that
\begin{align}
	A \leq B
\end{align}
we should also have
\begin{align}
	f(A) \leq f(B).
\end{align}

This kind of function might be more properly called \textit{matrix increasing} but we will mostly stick to the monotonicity for couple of reasons:
\begin{itemize}
	\item For some reason, that is what people have been doing in the field.
	\item It doesn't make much difference whether we talk about increasing or decreasing functions, so we might just ignore the latter but try to symmetrize our thinking by choice of words.
	\item Somehow I can't satisfactorily fill the following table:
	\begin{center}
	\begin{tabular}{| c | c |}
		\hline
		monotonic & monotonicity \\
		\hline
		increasing & ? \\
		\hline
	\end{tabular}
	\end{center}
	How very inconvenient.
\end{itemize}

Of course, it's not really obvious how one should make any sense of these ``definitions''. One quickly realizes that there two things to understand.
\begin{itemize}
	\item How should matrices be ordered?
	\item How should functions act on matrices?
\end{itemize}
Both of these questions can be (of course) answered in many ways, but for both of them, there is very natural answer. In both cases we can get something more general: instead of comparing matrices we can compare linear maps, and we can apply function to linear mapping.

Just to give a short glimpse of how these things might be defined, we should first fix our ground field (for matrices): let's say it's $\R$, at least for now.

For matrix ordering we should first understand which matrices are \textit{positive}, which here, a bit confusingly maybe, means ``at least zero". We can't have everything. For instance it's not very hard to see that it is not possible to give notion of positivity on space of all real $n \times n$ matrices which respects similarity. If we give our space inner product, and restrict to a nice subspace of linear maps, called Hermitian maps, we can have a notion of positivity which respects unitary similarity.

Matrix functions, i.e. ``how to apply function to matrix" is bit simpler to explain. Instead of doing something arbitrary the idea is to take real function (a function $f : \R \to \R$, say) and intepret it as function $f : \R^{n \times n} \to \R^{n \times n}$, \textit{matrix function}. Polynomials extend rather naturally, and similarly analytic functions, or at least entire. Now, a perverse definition for matrix function for continuous functions would be some kind of a limit when function is uniformly approximated by polynomials (using Weierstrass approximation theorem). This works for Hermitian matrices, but one can do better: apply the function to the eigenvalues of the mapping to get another linear map.

As it turns out, much of the study of matrix monotone and convex functions is all about understanding these definitions of positive maps and matrix functions.

Lastly, one might wonder why should one be interested in the whole business of matrix monotone functions? It's all about point of view. Let's consider a very simple inequality:

For any real numbers $0 < x \leq y$ we have
\[
	y^{-1} \leq x^{-1}.
\]
Of course, this is quite close to the axioms of the real numbers, but there's a rather fruitful interpretation. The function $(x \mapsto \frac{1}{x})$ is decreasing.

Now there's this matrix version of the previous inequality:

For any two matrices $0 < A \leq B$ we have
\[
	B^{-1} \leq A^{-1}.
\]
This is already not trivial, and with previous interpretation in mind, could this be interpreted as the functions $(x \mapsto \frac{1}{x})$ could be \textit{matrix decreasing}? And is this just a special case of something bigger? Yes, and that's exactly what this thesis is about.

\section{Themes}

The theory of matrix monotone functions is build upon two main themes.

\begin{enumerate}
	\item Positivity leads to boundedness, boundedness leads to regularity
	\item The bigger the space, the more ways are there to maneuver
\end{enumerate}

\section{Plan of attack}

This master's thesis is a comprehensive review of the rich theory of matrix monotone functions.

Master's thesis is to be structured roughly as follows.

\begin{enumerate}
	\item Introduction
		\begin{itemize}
			\item Introduction to the problem, motivation
			\item Brief definition of the matrix monotonicity and convexity
			\item Past and present (Is this the right place)
				\begin{itemize}
					\item Loewner's original work, Loewner-Heinz -inequality
					\item Students: Dobsch' and Krauss'
					\item Subsequent simplifications and further results: Bendat-Sherman, Wigner-Neumann, Koranyi, etc.
					\item Donoghue's work
					\item Later proofs: Krein-Milman, general spectral theorem, interpolation spaces, short proofs etc.
					\item Development of the convex case
					\item Recent simplifications, integral representations
					\item Operator inequalities
					\item Multivariate case, other variants
					\item Further open problems?
				\end{itemize}
			\item Scope of the thesis
		\end{itemize}
	\item Positive matrices
		\begin{itemize}
			\item Motivation via restriction, basics
			\item Spectral theorem
			\item Congruence
			\item Characterizations
			\item Applications
			\item Spectrum
		\end{itemize}
	\item Divided differences
		\begin{itemize}
			\item Definition (what kind of?)
			\item Mean value theorem
			\item Smoothness
			\item k-tone functions on $\R$
			\item Cauchy's integral formula
			\item Regularizations
		\end{itemize}
	\item Matrix functions
		\begin{itemize}
			\item Several definitions: spectral and cauchy
			\item Smoothness of matrix functions
		\end{itemize}
	\item Pick functions
		\begin{itemize}
			\item Basic definitions and properties
			\item Pick matrices/ determinants
			\item Compactness
			\item Pick-Nevanlinna interpolation theorem
			\item Pick-Nevanlinna representations theorem
		\end{itemize}
	\item Monotonic and convex matrix functions
		\begin{itemize}
			\item Basics
				\begin{itemize}
					\item Basic definitions and properties (cone structure, pointwise limits, compositions etc.)
					\item Classes $P_{n}, K_{n}$ and their properties
					\item $-1/x$
					\item One directions of Loewner's theorem
					\item Examples and non-examples
				\end{itemize}
			\item Pick matrices/determinants vs matrix monotone and convex functions
				\begin{itemize}
					\item Proofs for (sufficiently) smooth functions
				\end{itemize}
			\item Smoothness properties
				\begin{itemize}
					\item Ideas, simple cases
					\item General case by induction and regularizations
				\end{itemize}
			\item Global characterizations
				\begin{itemize}
					\item Putting everything together: we get original characterization of Loewner and determinant characterization
				\end{itemize}
		\end{itemize}
	\item Local characterizations
		\begin{itemize}
			\item Dobsch (Hankel) matrix: basic properties, easy direction (original and new proof)
			\item Integral representations
				\begin{itemize}
					\item Introducing the general weight functions for monotonicity and convexity (and beyond?)
					\item Non-negativity of the weights
					\item Proof of integral representations
				\end{itemize}
			\item Proof of local characterizations
		\end{itemize}
	\item Structure of the classes $P_{n}$ and $K_{n}$, interpolating properties (?)
		\begin{itemize}
			\item Strict inclusions, strict smoothness conditions
			\item Strictly increasing functions
			\item Extreme values
			\item Interpolating properties
		\end{itemize}
	\item Loewner's theorem
		\begin{itemize}
			\item Preliminary discussion, relation to operator monotone functions
			\item Loewner's original proof
			\item Pick-Nevanlinna proof
			\item Bendat-Sherman proof
			\item Krein-Milman proof
			\item Koranyi proof
			\item Discussion of the proofs
			\item Convex case
		\end{itemize}
	\item Alternative characterizations (?)
		\begin{itemize}
			\item Some discussion, maybe proofs
		\end{itemize}
	\item Bounded variations (?)
		\begin{itemize}
			\item Dobsch' definition, basic properties
			\item Decomposition, Dobsch' theorems
		\end{itemize}
\end{enumerate}

\section{How to rewrite this thesis}

\begin{enumerate}
	\item Positive maps: lose all the fat.
	\item Divided differences: concentrate on important things, namely relationship between smoothness and $k$-tone functions.
	\item Keep it relatively short, as it is (?)
	\item Pick functions: is this the place for these. Start with Schwarz lemma as an rigidity example. Then express Schwarz lemma with contour integrals: generalize, proof by tricks. Notion of Pick points, and finally Pick-Nevanlinna interpolation theorem, some form of it.
\end{enumerate}

\section{Some random ideas}
\begin{enumerate}
	\item TODO: fix Boor in the references
	\item It's easy to see that [Something]. Actually, it's so so easy that we have no excuse for not doing it.
	\item When is matrix of the form $f(a_{i} + a_{j})$ positive: $f$ is completely monotone (?).
	\item Polynomial regression...
	\item TODO: Maximum of two matrices (at least as big), $(a + b)/2 + abs(a - b)/2$
	\item If $\langle A x, y \rangle = 0$ implies $\langle x, A y \rangle = 0$, then $A$ is constant times hermitian.
	\item Angularity preserving functions
	\item If subspace of linear maps are diagonalizable with real eigenvalues, is there a inner product such that subspace consists of only Hermitian maps
	\item One should be alarmed should one see a positive cone.
	\item Make DAG (hopefully) of logical structure of the thesis, colour-coded (with respect to the topic, maybe). Theorem numbers, maybe named theorems with names. To the introduction.
	\item Cut the bullshit
\end{enumerate}

\section{Main TODO -list}
\subsection{Missing proofs}
\begin{enumerate}
	\item Eigenvalues of $AB$ when $A$ and $B$ are positive (?)
	\item Symmetric product fail (?)
	\item Hindmarsh theorem
	\item non-smooth Dobsch char.
	\item classes are different
	\item matrix $k$-tone (please expand)
\end{enumerate}

\subsection{Sections to write}
\begin{enumerate}
	\item \textcolor{red}{Integral representations}
	\item \textcolor{red}{All ``Notes and references" sections}
	\item More on convex and $k$-tone matrix functions
\end{enumerate}

\subsection{Figures to make}
\begin{enumerate}
	\item Proof of spectral theorem
	\item Compression
	\item Pictures of Peano kernels
	\item $k$-tone functions: visual definition
	\item $k$-tone functions are smooth
	\item Pick functions and measures
	\item Mean value theorem
	\item Pick extension lemma
	\item Eigenvalue inequalities: projections and compressions
	\item Concrete Pick function extension procedure
	\item Change of eigenvalues of $A + t B$
	\item Disc lemma
\end{enumerate}


