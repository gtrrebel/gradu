\chapter{Trace functions}

\section{Absolute Value}

As adjoint behaves as conjugate, it would be natural to guess that
\[
	|A| := \left(A^{*}A\right)^{\frac{1}{2}},
\]
absolute value of a map, would have many similar properties as the standard absolute value.

The following list of properties of the absolute value make it clear that this is indeed good definition.

\begin{itemize}
	\item $|A| \geq 0$ for any $A \in \L(V)$ and $|A| = A$, if and only if $A \geq 0$.
	\item For any $A \in \H(V)$ we have $-|A| \leq A \leq |A|$, or equivalently $|Q_{A}(v)| \leq Q_{|A|}(v)$ for any $v \in V$
	\item For any $v \in V$ we have $\|A v\| = \||A|v\|$.
\end{itemize}

 Note that in general we have $|A| \neq |A^{*}|$, and maps need not even go between the same spaces.

One might be tempted to think that we have triangle inequality, i.e.
\[
	|A + B| \leq |A| + |B|,
\]
for any $A, B \in \L(V)$, or at least $A, B \in \H$. Such inequality doesn't hold, but it's not that far from being true. Like in the real case, one would like to add
\[
	-|A| \leq A \leq |A| \; \text{ and } \; -|B| \leq B \leq |B|,
\]
to get
\[
	-(|A| + |B|) \leq A + B \leq |A| + |B|.
\]
The problem is that we can't make any further conclusions: just because $-Y \leq X \leq Y$, it is not necessarily the case that $|X| \leq Y$. Thinking in quadratic forms we get the inequality
\begin{align}
	|Q_{A + B}(v)| \leq Q_{|A| + |B|}(v),
\end{align}
for any $v \in V$, but this does not imply that $Q_{|A + B|}(v) \leq Q_{|A| + |B|}(v)$. Indeed $|Q_{A + B}(v)| \leq Q_{|A + B|}(v)$, as we noticed, so the inequality is going to the wrong direction. If however $v$ is an eigenvector of $A + B$, we have $|Q_{A + B}(v)| = Q_{|A + B|}(v)$, and it follows that
\[
	Q_{|A + B|}(v) \leq Q_{|A| + |B|}(v)
\]
holds for eigenvectors $v$ of $|A + B|$. Summing over the eigenvector we see that
\[
	\tr|A + B| \leq \tr |A| + \tr |B|,
\]
so instead of the full inequality, we get inequality for traces. There is a nice generalization for the previous we'll get back to.